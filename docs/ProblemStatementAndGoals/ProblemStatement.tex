\documentclass{article}

\usepackage{tabularx}
\usepackage{booktabs}

\title{Problem Statement and Goals\\\progname}

\author{\authname}

\date{}

%% Comments

\usepackage{color}

\newif\ifcomments\commentstrue %displays comments
%\newif\ifcomments\commentsfalse %so that comments do not display

\ifcomments
\newcommand{\authornote}[3]{\textcolor{#1}{[#3 ---#2]}}
\newcommand{\todo}[1]{\textcolor{red}{[TODO: #1]}}
\else
\newcommand{\authornote}[3]{}
\newcommand{\todo}[1]{}
\fi

\newcommand{\wss}[1]{\authornote{blue}{SS}{#1}} 
\newcommand{\plt}[1]{\authornote{magenta}{TPLT}{#1}} %For explanation of the template
\newcommand{\an}[1]{\authornote{cyan}{Author}{#1}}

%% Common Parts

\newcommand{\progname}{ProgName} % PUT YOUR PROGRAM NAME HERE
\newcommand{\authname}{Team \#, Team Name
\\ Student 1 name
\\ Student 2 name
\\ Student 3 name
\\ Student 4 name} % AUTHOR NAMES                  

\usepackage{hyperref}
    \hypersetup{colorlinks=true, linkcolor=blue, citecolor=blue, filecolor=blue,
                urlcolor=blue, unicode=false}
    \urlstyle{same}
                                


\begin{document}

\maketitle

\begin{table}[hp]
\caption{Revision History} \label{TblRevisionHistory}
\begin{tabularx}{\textwidth}{llX}
\toprule
\textbf{Date} & \textbf{Developer(s)} & \textbf{Change}\\
\midrule
Date1 & Name(s) & Description of changes\\
Date2 & Name(s) & Description of changes\\
... & ... & ...\\
\bottomrule
\end{tabularx}
\end{table}

\section{Problem Statement}

\subsection{Problem}

Dairy farmers face significant challenges in managing the health, productivity, and breeding outcomes of their herds. Traditionally, these issues are managed reactively, with farmers responding to problems such as health declines, breeding failures, or unexpected herd turnover after they occur. This reactive approach leads to inefficiencies, financial losses, and reduced herd performance. Our project, in collaboration with CATTLEytics Inc., aims to address these challenges by implementing a machine learning (ML) model capable of predicting breeding success rates and the likelihood of animals leaving the herd based on historical data. This model will help farmers transition from reactive to proactive herd management, improving overall farm productivity and decision-making. The model will integrate with CATTLEytics Inc.'s existing systems, allowing seamless adoption by farmers.

% \wss{You should check your problem statement with the
% \href{https://github.com/smiths/capTemplate/blob/main/docs/Checklists/ProbState-Checklist.pdf}
% {problem statement checklist}.} 

% \wss{You can change the section headings, as long as you include the required
% information.}



\subsection{Inputs and Outputs}

\paragraph{Inputs} \ \\
\\
The software will use historical herd data related to health, productivity, and breeding outcomes. This data may include metrics such as individual animal health records, breeding attempts, and farm-specific environmental conditions. The input data will be fed into a machine learning model to generate predictions.

\paragraph{Outputs} \ \\
\\
The output will consist of predictive insights aimed at improving farm management. These will include:
\begin{enumerate}
    \item Predictions about the success of breeding attempts for individual animals.
    \item The likelihood of an animal leaving the herd.
    \item Alerts about potential health issues or productivity declines.
\end{enumerate}

The software will provide clear, actionable outputs to guide decision-making, presented in a format easily interpretable by farmers and other stakeholders.



\subsection{Stakeholders}

\begin{enumerate}
    \item \textbf{Farmers:} The primary users of the software, who will benefit from improved herd management and decision-making.
    \item \textbf{CATTLEytics Inc.:} The company that will deploy the software to its customers and provide ongoing support.
    \item \textbf{Animal Health Experts:} Professionals who may use the software to provide advice to farmers based on the predictive insights generated.
    \item \textbf{Regulatory Bodies:} Organizations that may use the software to monitor herd health and productivity.
    \item \textbf{Dairy Production Companies:} Organizations that may use the software to optimize their supply chain and production processes.
\end{enumerate}


\subsection{Environment}

\paragraph{Hardware} \ \\
\\
The software will be deployed in farm management systems typically used on standard personal computers or servers. These systems may also interact with IoT devices used for real-time data collection on farms.
\paragraph{Software} \ \\
\\
The model will be integrated into CATTLEytics Inc.'s farm management platform, utilizing their existing infrastructure. The model will run within a cloud-based or on-premise system, ensuring smooth integration into the farmers' existing workflow, while maintaining compatibility with the tools they are already using.

\section{Product Goals and Stretch Goals}

\begin{table}[!htbp]
\centering
\caption{Product Goals and Stretch Goals}

% Product Goals Table
\begin{tabularx}{\textwidth}{|X|X|}
\hline
\textbf{Goal} & \textbf{Importance} \\ \hline
The model accurately predicts breeding success rates for individual cows. & This is the core function of the model. Accurate predictions will help farmers improve their breeding programs and overall herd productivity. \\ \hline
The model provides predictions on the likelihood of an animal leaving the herd. & This allows proactive herd management, helping farmers reduce turnover and improve herd stability. \\ \hline
The system integrates seamlessly into CATTLEytics Inc.'s existing platform. & Integration into CATTLEytics Inc. ensures that farmers can easily adopt the model without disrupting their current workflow. \\ \hline
The system offers user-friendly, actionable insights for farmers. & Clear, easy-to-understand outputs will help farmers make informed decisions without needing technical expertise. \\ \hline
The model processes real-time data to provide up-to-date predictions. & Real-time predictions allow farmers to respond quickly to changes in herd dynamics, preventing potential issues. \\ \hline
\end{tabularx}

\vspace{5mm} % Adjust this space to prevent overlapping with the next table

% Stretch Goals Table
\begin{tabularx}{\textwidth}{|X|X|}
\hline
\textbf{Goal} & \textbf{Importance} \\ \hline
The system includes a feature for predicting long-term health trends of individual cows. & This would enable farmers to anticipate health problems and intervene early, leading to better animal welfare and reduced veterinary costs. \\ \hline
The model can be customized for different farm sizes and breeds. & Customization increases the model's marketability, allowing it to be adapted to various farm environments and specific herd characteristics. \\ \hline
The system integrates renewable energy sources for data processing and storage. & Renewable energy use aligns with sustainability goals and makes the system more appealing to environmentally-conscious customers. \\ \hline
The system offers encrypted data storage and transmission to ensure data security. & Data security is crucial for farmers and companies that handle sensitive herd information. This feature increases trust and protects data integrity. \\ \hline
\end{tabularx}

\end{table}


\section{Challenge Level and Extras}

\wss{State your expected challenge level (advanced, general or basic).  The
challenge can come through the required domain knowledge, the implementation or
something else.  Usually the greater the novelty of a project the greater its
challenge level.  You should include your rationale for the selected level.
Approval of the level will be part of the discussion with the instructor for
approving the project.  The challenge level, with the approval (or request) of
the instructor, can be modified over the course of the term.}

\wss{Teams may wish to include extras as either potential bonus grades, or to
make up for a less advanced challenge level.  Potential extras include usability
testing, code walkthroughs, user documentation, formal proof, GenderMag
personas, Design Thinking, etc.  Normally the maximum number of extras will be
two.  Approval of the extras will be part of the discussion with the instructor
for approving the project.  The extras, with the approval (or request) of the
instructor, can be modified over the course of the term.}

\newpage{}

\section*{Appendix --- Reflection}

\wss{Not required for CAS 741}

The purpose of reflection questions is to give you a chance to assess your own
learning and that of your group as a whole, and to find ways to improve in the
future. Reflection is an important part of the learning process.  Reflection is
also an essential component of a successful software development process.  

Reflections are most interesting and useful when they're honest, even if the
stories they tell are imperfect. You will be marked based on your depth of
thought and analysis, and not based on the content of the reflections
themselves. Thus, for full marks we encourage you to answer openly and honestly
and to avoid simply writing ``what you think the evaluator wants to hear.''

Please answer the following questions.  Some questions can be answered on the
team level, but where appropriate, each team member should write their own
response:


\begin{enumerate}
    \item What went well while writing this deliverable? 
    \item What pain points did you experience during this deliverable, and how
    did you resolve them?
    \item How did you and your team adjust the scope of your goals to ensure
    they are suitable for a Capstone project (not overly ambitious but also of
    appropriate complexity for a senior design project)?
\end{enumerate}  

\end{document}