\documentclass{article}

\usepackage{booktabs}
\usepackage{tabularx}
\usepackage{hyperref}

\hypersetup{
    colorlinks=true,       % false: boxed links; true: colored links
    linkcolor=red,          % color of internal links (change box color with linkbordercolor)
    citecolor=green,        % color of links to bibliography
    filecolor=magenta,      % color of file links
    urlcolor=cyan           % color of external links
}

\title{Hazard Analysis\\\progname}

\author{\authname}

\date{}

%% Comments

\usepackage{color}

\newif\ifcomments\commentstrue %displays comments
%\newif\ifcomments\commentsfalse %so that comments do not display

\ifcomments
\newcommand{\authornote}[3]{\textcolor{#1}{[#3 ---#2]}}
\newcommand{\todo}[1]{\textcolor{red}{[TODO: #1]}}
\else
\newcommand{\authornote}[3]{}
\newcommand{\todo}[1]{}
\fi

\newcommand{\wss}[1]{\authornote{blue}{SS}{#1}} 
\newcommand{\plt}[1]{\authornote{magenta}{TPLT}{#1}} %For explanation of the template
\newcommand{\an}[1]{\authornote{cyan}{Author}{#1}}

%% Common Parts

\newcommand{\progname}{ProgName} % PUT YOUR PROGRAM NAME HERE
\newcommand{\authname}{Team \#, Team Name
\\ Student 1 name
\\ Student 2 name
\\ Student 3 name
\\ Student 4 name} % AUTHOR NAMES                  

\usepackage{hyperref}
    \hypersetup{colorlinks=true, linkcolor=blue, citecolor=blue, filecolor=blue,
                urlcolor=blue, unicode=false}
    \urlstyle{same}
                                


\begin{document}

\maketitle
\thispagestyle{empty}

~\newpage

\pagenumbering{roman}

\begin{table}[hp]
\caption{Revision History} \label{TblRevisionHistory}
\begin{tabularx}{\textwidth}{llX}
\toprule
\textbf{Date} & \textbf{Developer(s)} & \textbf{Change}\\
\midrule
Date1 & Name(s) & Description of changes\\
Date2 & Name(s) & Description of changes\\
... & ... & ...\\
\bottomrule
\end{tabularx}
\end{table}

~\newpage

\tableofcontents

~\newpage

\pagenumbering{arabic}

\wss{You are free to modify this template.}

\section{Introduction}

\wss{You can include your definition of what a hazard is here.}

\section{Scope and Purpose of Hazard Analysis}

\wss{You should say what \textbf{loss} could be incurred because of the
hazards.}

\section{System Boundaries and Components}

\wss{Dividing the system into components will help you brainstorm the hazards.
You shouldn't do a full design of the components, just get a feel for the major
ones.  For projects that involve hardware, the components will typically include
each individual piece of hardware.  If your software will have a database, or an
important library, these are also potential components.}

\section{Critical Assumptions}

This section outlines the key assumptions made about the system and its 
operating environment. These assumptions are intended to define the boundaries 
and to manage potential risks.\\

\textbf{Data Collection Limitations} 
\begin{itemize} 
    \item \textbf{Assumption:} All data used for model training and predictions 
    will be sourced from existing datasets provided by CATTLEytics Inc. or 
    relevant partners. No additional data collection will be undertaken by the 
    project team. 
    \item \textbf{Rationale:} The scope of the project does not include new data 
    collection activities, limiting the system to work with available historical 
    and provided real-time data. This constraint may limit the model's ability to 
    adapt to emerging conditions not represented in the existing dataset. 
    \item \textbf{Mitigation:} Ensure a comprehensive understanding of the provided 
    dataset's limitations, and apply model validation techniques 
    (such as cross-validation) to mitigate risks associated with potential 
    biases or gaps in the data. Regular updates to the dataset from the provider 
    should also be encouraged to keep the model relevant. 
\end{itemize}

\textbf{Data Accuracy and Availability}
\begin{itemize}
    \item \textbf{Assumption:} The historical and real-time data provided to the 
    machine learning model are accurate, complete, and relevant to the cows and 
    their environment.
    \item \textbf{Rationale:} The model's accuracy and reliability depend heavily 
    on the quality of input data. Inaccurate or incomplete data could lead to 
    faulty predictions and management decisions.
    \item \textbf{Mitigation:} Implement data validation procedures to ensure 
    the accuracy and completeness of incoming data.
\end{itemize}

\textbf{Stable Operating Conditions}
\begin{itemize}
    \item \textbf{Assumption:} The environmental and operational conditions on 
    the farm (such as temperature, feed availability, and animal health 
    monitoring) remain within expected ranges during the model's operation.
    \item \textbf{Rationale:} Significant deviations in farm conditions could 
    affect the model's performance, leading to inaccurate predictions.
    \item \textbf{Mitigation:} Develop contingency measures or notification 
    systems for when conditions fall outside normal ranges.
\end{itemize}

\textbf{Model Generalizability}
\begin{itemize}
    \item \textbf{Assumption:} The machine learning model generalizes well 
    across different farms and herd conditions without needing extensive 
    retraining for every farm.
    \item \textbf{Rationale:} The model is intended to be a broadly applicable 
    tool across multiple farms with varying conditions. Excessive reliance on 
    farm-specific parameters would increase complexity and reduce scalability.
    \item \textbf{Mitigation:} Incorporate regular model updates and feedback 
    loops to address variability between different environments.
\end{itemize}

\textbf{Ethical Use and Compliance}
\begin{itemize}
    \item \textbf{Assumption:} Users will adhere to legal and ethical standards 
    when implementing recommendations made by the tool, particularly regarding 
    animal welfare and data privacy.
    \item \textbf{Rationale:} Legal compliance (e.g., PIPEDA) and ethical 
    considerations (animal welfare) are critical to ensuring the tool is used 
    responsibly.
    \item \textbf{Mitigation:} Provide user education and warnings for critical 
    decisions that impact animal welfare, and ensure compliance with data privacy laws.
\end{itemize}

\textbf{Technical Infrastructure Reliability}
\begin{itemize}
    \item \textbf{Assumption:} The technical infrastructure (servers, farm 
    management platforms, IoT devices) that supports the model operates 
    reliably with minimal downtime or technical failures.
    \item \textbf{Rationale:} System malfunctions or data transmission failures 
    could lead to gaps in model predictions, causing delays in critical 
    decision-making.
    \item \textbf{Mitigation:} Implement fault-tolerance mechanisms and 
    regular system health checks to ensure the infrastructure remains stable.
\end{itemize}


\section{Failure Mode and Effect Analysis}

\wss{Include your FMEA table here. This is the most important part of this document.}
\wss{The safety requirements in the table do not have to have the prefix SR.
The most important thing is to show traceability to your SRS. You might trace to
requirements you have already written, or you might need to add new
requirements.}
\wss{If no safety requirement can be devised, other mitigation strategies can be
entered in the table, including strategies involving providing additional
documentation, and/or test cases.}

\section{Safety and Security Requirements}

\wss{Newly discovered requirements.  These should also be added to the SRS.  (A
rationale design process how and why to fake it.)}

\section{Roadmap}

\wss{Which safety requirements will be implemented as part of the capstone timeline?
Which requirements will be implemented in the future?}

\newpage{}

\section*{Appendix --- Reflection}

\wss{Not required for CAS 741}

The purpose of reflection questions is to give you a chance to assess your own
learning and that of your group as a whole, and to find ways to improve in the
future. Reflection is an important part of the learning process.  Reflection is
also an essential component of a successful software development process.  

Reflections are most interesting and useful when they're honest, even if the
stories they tell are imperfect. You will be marked based on your depth of
thought and analysis, and not based on the content of the reflections
themselves. Thus, for full marks we encourage you to answer openly and honestly
and to avoid simply writing ``what you think the evaluator wants to hear.''

Please answer the following questions.  Some questions can be answered on the
team level, but where appropriate, each team member should write their own
response:


\begin{enumerate}
    \item What went well while writing this deliverable? 
    \item What pain points did you experience during this deliverable, and how
    did you resolve them?
    \item Which of your listed risks had your team thought of before this
    deliverable, and which did you think of while doing this deliverable? For
    the latter ones (ones you thought of while doing the Hazard Analysis), how
    did they come about?
    \item Other than the risk of physical harm (some projects may not have any
    appreciable risks of this form), list at least 2 other types of risk in
    software products. Why are they important to consider?
\end{enumerate}

\end{document}