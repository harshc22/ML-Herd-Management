\documentclass{article}

\usepackage{booktabs}
\usepackage{tabularx}
\usepackage{hyperref}

\hypersetup{
    colorlinks=true,       % false: boxed links; true: colored links
    linkcolor=red,          % color of internal links (change box color with linkbordercolor)
    citecolor=green,        % color of links to bibliography
    filecolor=magenta,      % color of file links
    urlcolor=cyan           % color of external links
}

\title{Hazard Analysis\\\progname}

\author{\authname}

\date{}

%% Comments

\usepackage{color}

\newif\ifcomments\commentstrue %displays comments
%\newif\ifcomments\commentsfalse %so that comments do not display

\ifcomments
\newcommand{\authornote}[3]{\textcolor{#1}{[#3 ---#2]}}
\newcommand{\todo}[1]{\textcolor{red}{[TODO: #1]}}
\else
\newcommand{\authornote}[3]{}
\newcommand{\todo}[1]{}
\fi

\newcommand{\wss}[1]{\authornote{blue}{SS}{#1}} 
\newcommand{\plt}[1]{\authornote{magenta}{TPLT}{#1}} %For explanation of the template
\newcommand{\an}[1]{\authornote{cyan}{Author}{#1}}

%% Common Parts

\newcommand{\progname}{ProgName} % PUT YOUR PROGRAM NAME HERE
\newcommand{\authname}{Team \#, Team Name
\\ Student 1 name
\\ Student 2 name
\\ Student 3 name
\\ Student 4 name} % AUTHOR NAMES                  

\usepackage{hyperref}
    \hypersetup{colorlinks=true, linkcolor=blue, citecolor=blue, filecolor=blue,
                urlcolor=blue, unicode=false}
    \urlstyle{same}
                                


\begin{document}

\maketitle
\thispagestyle{empty}

~\newpage

\pagenumbering{roman}

\begin{table}[hp]
\caption{Revision History} \label{TblRevisionHistory}
\begin{tabularx}{\textwidth}{llX}
\toprule
\textbf{Date} & \textbf{Developer(s)} & \textbf{Change}\\
\midrule
October 20th, 2024 & Harsh Chinjer & Introduction, Scope and Purpose\\
Date2 & Name(s) & Description of changes\\
... & ... & ...\\
\bottomrule
\end{tabularx}
\end{table}

~\newpage

\tableofcontents

~\newpage

\pagenumbering{arabic}

\wss{You are free to modify this template.}

\section{Introduction}

This project involves developing a machine learning tool in collaboration with
Cattleytics Inc., aimed at predicting key outcomes for dairy farming. The tool
will help dairy farmers make informed decisions regarding breeding, herd
management, and productivity by analyzing historical and genetic data. By
predicting traits such as milk output, health conditions, and breeding success,
the project strives to optimize herd performance and improve overall farm
management.

A hazard in this context refers to any event, condition, or system failure that
could result in negative consequences such as financial losses, breaches of
ethical standards, or harm to livestock. Hazards can arise from various factors,
including data inaccuracies, mispredictions from the model, technical
malfunctions, or security breaches, all of which may impair decision-making.
Since the system is designed to influence critical choices—like selecting sires
or predicting future traits—it is vital to identify and address potential
hazards to ensure the tool’s reliability, ethical operation, and overall benefit
to dairy farming.

\section{Scope and Purpose of Hazard Analysis}

The purpose of this hazard analysis is to systematically identify, assess, and
mitigate potential risks associated with the machine learning model and its
integration into the broader farm management system (Cattleytics). The primary
scope includes the hazards that could affect the accuracy and reliability of the
predictions, the security of sensitive farmer and animal data, and the ethical
handling of dairy cattle based on the recommendations generated by the tool.

The potential losses that could be incurred due to these hazards include:

\begin{itemize}
	\item Financial Loss: If the model makes incorrect predictions regarding
	      breeding success rates, milk production, or herd turnover, farmers
	      could experience financial setbacks, such as poor breeding decisions
	      leading to reduced milk yield or ineffective herd management.
	      
	\item Reputation Damage: Incorrect or biased predictions could lead to poor
	      decision-making, negatively affecting the credibility of both the tool
	      and the farmers who rely on it. Additionally, improper handling of
	      personal information could damage trust with stakeholders.
	      
	\item Legal Repercussions: Non-compliance with the Code of Practice for the
	      Care and Handling of Dairy Cattle or PIPEDA could result in legal
	      action or fines for mishandling sensitive data or violating animal
	      welfare standards.
	      
	\item Ethical Concerns: A failure to comply with the ethical treatment
	      standards of livestock could lead to the mistreatment of cattle, which
	      would not only result in legal issues but also ethical concerns around
	      animal welfare.
	      
\end{itemize}

This hazard analysis aims to identify such risks early in the project, allowing
for proactive strategies to minimize harm and ensure the safe, secure, and
effective operation of the tool in dairy farming.

\section{System Boundaries and Components}

\wss{Dividing the system into components will help you brainstorm the hazards.
You shouldn't do a full design of the components, just get a feel for the major
ones.  For projects that involve hardware, the components will typically include
each individual piece of hardware.  If your software will have a database, or an
important library, these are also potential components.}

\section{Critical Assumptions}

\wss{These assumptions that are made about the software or system.  You should
minimize the number of assumptions that remove potential hazards.  For instance,
you could assume a part will never fail, but it is generally better to include
this potential failure mode.}

\section{Failure Mode and Effect Analysis}

\wss{Include your FMEA table here. This is the most important part of this document.}
\wss{The safety requirements in the table do not have to have the prefix SR.
The most important thing is to show traceability to your SRS. You might trace to
requirements you have already written, or you might need to add new
requirements.}
\wss{If no safety requirement can be devised, other mitigation strategies can be
entered in the table, including strategies involving providing additional
documentation, and/or test cases.}

\section{Safety and Security Requirements}

\subsection{Safety Requirements}

\textbf{SFR1: Prevent Inbreeding Between Sire and Dam} 
\begin{itemize}
    \item \textbf{Description}: The system must prevent inbreeding by detecting 
    when a sire and dam are closely related and disallowing or flagging those 
    breeding pairs. The system should calculate the relationship coefficient 
    between breeding pairs and raise an alert if it exceeds a predefined 
    threshold.
    \item \textbf{Rationale}: Inbreeding increases the risk of genetic 
    disorders, reduced fertility, and poor overall herd health. Preventing 
    inbreeding ensures better genetic diversity and healthier offspring.
    \item \textbf{Fit Criterion}: The system should compare genetic records of 
    sires and dams, ensuring that no breeding pair with a relationship 
    coefficient above a set threshold is approved. This value can currently set 
    to 10\%, but will likely to re-evaluated later. A warning or error message 
    must be displayed for all invalid breeding pairs, with 100\% accuracy in 
    flagging related pairs during tests.
\end{itemize}
\vspace{10pt}

\subsection{Security Requirements}

\textbf{SCR1: Backup and Recovery Mechanism} 
\begin{itemize}
    \item \textbf{Description}: The system must regularly back up its data to 
    ensure that critical information, such as local breeding data, developed models, and so forth are not lost in case of system failure or crash. It should also 
    provide a reliable recovery mechanism to restore data after an incident.
    \item \textbf{Rationale}: Data loss can lead to incorrect predictions, poor 
    breeding decisions, and financial losses for farmers. A backup and recovery 
    mechanism ensures that the system remains reliable and that users can 
    recover from potential failures.
    \item \textbf{Fit Criterion}: The system must successfully back up data 
    every 24 hours, witmah at least 99\% success in restoring data during tests.
\end{itemize}
\vspace{10pt}

\textbf{SCR2: Multi-Factor Authentication (MFA)} 
\begin{itemize}
    \item \textbf{Description}: The system must enforce multi-factor 
    authentication for all administrative users to ensure secure access to 
    sensitive data, such as genetic and breeding records. MFA should require 
    users to authenticate using two or more credentials, such as passwords and 
    authentication tokens.
    \item \textbf{Rationale}: Sensitive data requires an additional layer of 
    security to prevent unauthorized access. By using MFA, the risk of data 
    breaches and unauthorized access is minimized.
    \item \textbf{Fit Criterion}: MFA must be implemented and enforced for all 
    administrative-level accounts, with at least 95\% of users successfully 
    authenticating during tests.
\end{itemize}

\section{Roadmap}

SFR1 will be included in the capstone timeline as it involves safety and 
ethical concerns. It will be implemented when the system is being built. The 
other safety concerns are not of high priority and will therefore may be 
inlcuded if there is extra time.

\newpage{}

\section*{Appendix --- Reflection}

\wss{Not required for CAS 741}

The purpose of reflection questions is to give you a chance to assess your own
learning and that of your group as a whole, and to find ways to improve in the
future. Reflection is an important part of the learning process.  Reflection is
also an essential component of a successful software development process.  

Reflections are most interesting and useful when they're honest, even if the
stories they tell are imperfect. You will be marked based on your depth of
thought and analysis, and not based on the content of the reflections
themselves. Thus, for full marks we encourage you to answer openly and honestly
and to avoid simply writing ``what you think the evaluator wants to hear.''

Please answer the following questions.  Some questions can be answered on the
team level, but where appropriate, each team member should write their own
response:


\begin{enumerate}
    \item What went well while writing this deliverable? 
    \item What pain points did you experience during this deliverable, and how
    did you resolve them?
    \item Which of your listed risks had your team thought of before this
    deliverable, and which did you think of while doing this deliverable? For
    the latter ones (ones you thought of while doing the Hazard Analysis), how
    did they come about?
    \item Other than the risk of physical harm (some projects may not have any
    appreciable risks of this form), list at least 2 other types of risk in
    software products. Why are they important to consider?
\end{enumerate}

\end{document}


