\documentclass{article}

\usepackage{booktabs}
\usepackage{tabularx}
\usepackage{enumitem}

\title{Development Plan\\\progname}

\author{\authname}

\date{}

%% Comments

\usepackage{color}

\newif\ifcomments\commentstrue %displays comments
%\newif\ifcomments\commentsfalse %so that comments do not display

\ifcomments
\newcommand{\authornote}[3]{\textcolor{#1}{[#3 ---#2]}}
\newcommand{\todo}[1]{\textcolor{red}{[TODO: #1]}}
\else
\newcommand{\authornote}[3]{}
\newcommand{\todo}[1]{}
\fi

\newcommand{\wss}[1]{\authornote{blue}{SS}{#1}} 
\newcommand{\plt}[1]{\authornote{magenta}{TPLT}{#1}} %For explanation of the template
\newcommand{\an}[1]{\authornote{cyan}{Author}{#1}}

%% Common Parts

\newcommand{\progname}{ProgName} % PUT YOUR PROGRAM NAME HERE
\newcommand{\authname}{Team \#, Team Name
\\ Student 1 name
\\ Student 2 name
\\ Student 3 name
\\ Student 4 name} % AUTHOR NAMES                  

\usepackage{hyperref}
    \hypersetup{colorlinks=true, linkcolor=blue, citecolor=blue, filecolor=blue,
                urlcolor=blue, unicode=false}
    \urlstyle{same}
                                


\begin{document}

\maketitle

\begin{table}[hp]
\caption{Revision History} \label{TblRevisionHistory}
\begin{tabularx}{\textwidth}{llX}
\toprule
\textbf{Date} & \textbf{Developer(s)} & \textbf{Change}\\
\midrule
September 22 2024 & Martin Ivanov & Added team member roles and POC demonstration plan\\
September 22 2024 & Harshpreet Chinjer & Added meeting plan, Expected technology and Coding standard\\
... & ... & ...\\
\bottomrule
\end{tabularx}
\end{table}

\newpage{}

\wss{Put your introductory blurb here.  Often the blurb is a brief roadmap of
what is contained in the report.}

\wss{Additional information on the development plan can be found in the
\href{https://gitlab.cas.mcmaster.ca/courses/capstone/-/blob/main/Lectures/L02b_POCAndDevPlan/POCAndDevPlan.pdf?ref_type=heads}
{lecture slides}.}

\section{Confidential Information?}

\wss{State whether your project has confidential information from industry, or
not.  If there is confidential information, point to the agreement you have in
place.}

\wss{For most teams this section will just state that there is no confidential
information to protect.}
\section{IP to Protect}

\wss{State whether there is IP to protect.  If there is, point to the agreement.
All students who are working on a project that requires an IP agreement are also
required to sign the ``Intellectual Property Guide Acknowledgement.''}

\section{Copyright License}

\wss{What copyright license is your team adopting.  Point to the license in your
repo.}

\section{Team Meeting Plan}

\begin{enumerate}[label=\textbf{\arabic*}]
    \item \textbf{Frequency of Meetings}
    \begin{itemize}
        \item \textbf{Team Meetings}: The team will meet virtually once a week at discussed time, with in-person meetings if required
        \item \textbf{Industry Supervisor Meetings}: Meetings with the industry supervisor will occur virtually once a week, with additional meetings or cancellations based on the supervisor's availability and project progress.
        \item \textbf{Additional Meetings}: The team or supervisor may call extra meetings as needed, especially during critical project milestones.
    \end{itemize}
    
    \item \textbf{Meeting Location}
    \begin{itemize}
        \item \textbf{Virtual Meetings}: Conducted via platforms Discord or Microsoft Teams.
        \item \textbf{In-Person Meetings}: Arranged at a mutually agreed location like Mills Memorial Library, H.G. Thode Library and ITB Hallway Space 
    \end{itemize}

    \item \textbf{Meeting Structure}
    \begin{itemize}
    \item \textbf{Chair Rotation}: Each team member will take turns chairing the meetings. This includes leading discussions, ensuring the agenda is followed, and managing time.
    \item \textbf{Agenda}: The chair of the upcoming meeting is responsible for preparing the agenda and circulating it to all members at least \textbf{24 hours prior} to the meeting. The agenda will include:
    \begin{itemize}
        \item Project updates
        \item Task allocation
        \item Deadlines review
        \item Challenges and risks
        \item Next steps
        \item Supervisor queries (for meetings with the industry advisor)
    \end{itemize}
    \end{itemize}

    \item \textbf{Meeting Protocol}
    \begin{itemize}
        \item \textbf{Attendance}: All members are expected to attend unless unable, in which case they should inform the members ahead of time.
        \item \textbf{Minutes}: One team member will be assigned to take minutes, including key decisions, action items, and next steps. Minutes will be shared within \textbf{24 hours} after the meeting on GitHub.
    \end{itemize}

\end{enumerate}


\section{Team Communication Plan}

\wss{Issues on GitHub should be part of your communication plan.}

\section{Team Member Roles}

\wss{You should identify the types of roles you anticipate, like notetaker,
leader, meeting chair, reviewer.  Assigning specific people to those roles is
not necessary at this stage.  In a student team the role of the individuals will
likely change throughout the year.}

\subsection*{Liaison} Responsible for communicating with the industry supervisor, ensuring that the project is on track, and addressing any concerns or questions from the supervisor.   
\subsection*{Chair} Responsible for leading meetings, ensuring the agenda is followed, and managing time. Team members will take turns being the chair for each meeting.
\subsection*{Meeting Minutes} Responsible for taking minutes during meetings, including key decisions, action items, and next steps. Meeting minutes will be taken by a single member at a time, and the member taking minutes will rotate every week.
\subsection*{Machine Learning Lead} Responsible for leading the machine learning component of the project, including researching machine learning models, and ensuring that the machine learning component meets the requirements.
\subsection*{Developer} Responsible for developing the project, including coding, testing, and documentation. Developers must ensure that the code produced for the project is of high quality and is well-documented.
\subsection*{Tester} Responsible for testing the project, including unit testing, integration testing, and system testing. This role will be responsible for ensuring that the project meets the requirements and is free of defects.
\subsection*{Path to Changing Roles} Team members can change roles based on their interest, availability, and skill set. The team will discuss and decide on role changes as needed. Furthermore, any team member may be asked to help out anywhere else in the system if the need arises. For this reason, it is important that all team members stay up to date with context for what is happening with different parts of the team. Team members may fulfill more than one role at once.

\section{Workflow Plan}

\begin{itemize}
	\item How will you be using git, including branches, pull request, etc.?
	\item How will you be managing issues, including template issues, issue
	classification, etc.?
  \item Use of CI/CD
\end{itemize}

\section{Project Decomposition and Scheduling}

\begin{itemize}
  \item How will you be using GitHub projects?
  \item Include a link to your GitHub project
\end{itemize}

\wss{How will the project be scheduled?  This is the big picture schedule, not
details. You will need to reproduce information that is in the course outline
for deadlines.}

\section{Proof of Concept Demonstration Plan}

A few potential challenges that may arise during the course of this project include the availability and quality of historical dairy farm data. While we will be provided with some datasets related to animal health, breeding success, and productivity, these may suffer from inconsistencies, missing values, or lack of relevant features. Our strategy to overcome this involves performing thorough data preprocessing steps to ensure the dataset is suitable for building a reliable model. Additionally, the complexity of predicting outcomes like breeding success or the likelihood of an animal leaving the herd poses a challenge in terms of model accuracy. To address this, we plan to conduct some experiments using various machine learning algorithms to identify the most accurate and robust model for these predictions.

For our POC demonstration in November, we will build a prototype model using a limited dataset that predicts breeding success rates. The POC will include a data pipeline that handles preprocessing steps, a prediction model that outputs breeding success probabilities, and means for displaying the predictions via a dashboard or command line output. This will allow us to demonstrate the viability of the system and our ability to overcome challenges related to data quality and prediction accuracy. In the final product, the model will be scaled to handle more complex data and generate additional insights such as health risks or productivity forecasts, but if the POC can show accurate predictions for a small subset of outcomes, it serves as proof that the system will work for more complex scenarios.

\section{Expected Technology}
\textbf{Initial Implementation Plan}

\begin{itemize}
    \item \textbf{Programming language}: 
    The project will primarily use \textbf{Python} due to its strong support for machine learning and data science tasks. For web development tasks, the project will use \textbf{React}, a \textbf{JavaScript} library. 
    \item \textbf{Libraries}: 
    Expected key libraries include \textbf{Pandas} and \textbf{Numpy} for data manipulation, \textbf{Scikit-learn}, \textbf{TensorFlow}, or \textbf{PyTorch} for building the machine learning model, and \textbf{Matplotlib} or \textbf{Seaborn} for data visualization.
    \item \textbf{AI Model}: 
    The project will create a \textbf{custom AI model} specifically tailored to dairy farming data, focusing on predicting outcomes such as breeding success rates and herd attrition.
    \item \textbf{Linter}: 
    \textbf{Flake8} or \textbf{Black} will be utilized to maintain code quality and ensure adherence to Python's PEP 8 standards.
    \item \textbf{Unit testing framework}: 
    \textbf{PyTest} will be used to implement unit tests, focusing on validating data pipelines, model training, and prediction accuracy.
    \item \textbf{Continuous Integration (CI)}: 
    \textbf{GitHub Actions} will be used to automate tests and ensure code quality in a continuous integration pipeline.
    \item \textbf{Version Control and Project Management}: 
    \textbf{Git} will be used for version control and \textbf{GitHub} for repository management and collaboration throughout the project.
\end{itemize}

\section{Coding Standard}

The project will follow the \textbf{PEP 8} coding standard for Python, which is the official style guide for Python code. PEP 8 outlines guidelines and best practices for writing clean, readable, and maintainable Python code. Key recommendations include:

\begin{itemize}
    \item Using 4 spaces per indentation level.
    \item Limiting line length to 79 characters.
    \item Proper use of blank lines to separate functions and classes.
    \item Consistent use of lower\_case\_with\_underscores for variable and function names.
    \item Avoiding extraneous whitespace.
    \item Using meaningful comments to explain the purpose of code blocks.
    \item Keeping code simple and readable, and adhering to Python’s philosophy of clarity.
    
\end{itemize}

For a full reference to PEP 8 guidelines, visit the official document: \href{https://peps.python.org/pep-0008/}{PEP 8 Style Guide}.


\newpage{}

\section*{Appendix --- Reflection}

\wss{Not required for CAS 741}

The purpose of reflection questions is to give you a chance to assess your own
learning and that of your group as a whole, and to find ways to improve in the
future. Reflection is an important part of the learning process.  Reflection is
also an essential component of a successful software development process.  

Reflections are most interesting and useful when they're honest, even if the
stories they tell are imperfect. You will be marked based on your depth of
thought and analysis, and not based on the content of the reflections
themselves. Thus, for full marks we encourage you to answer openly and honestly
and to avoid simply writing ``what you think the evaluator wants to hear.''

Please answer the following questions.  Some questions can be answered on the
team level, but where appropriate, each team member should write their own
response:


\begin{enumerate}
    \item Why is it important to create a development plan prior to starting the
    project?
    \item In your opinion, what are the advantages and disadvantages of using
    CI/CD?
    \item What disagreements did your group have in this deliverable, if any,
    and how did you resolve them?
\end{enumerate}

\newpage{}

\section*{Appendix --- Team Charter}

\wss{borrows from
\href{https://engineering.up.edu/industry_partnerships/files/team-charter.pdf}
{University of Portland Team Charter}}

\subsection*{External Goals}

\wss{What are your team's external goals for this project? These are not the
goals related to the functionality or quality fo the project.  These are the
goals on what the team wishes to achieve with the project.  Potential goals are
to win a prize at the Capstone EXPO, or to have something to talk about in
interviews, or to get an A+, etc.}

\subsection*{Attendance}

\subsubsection*{Expectations}

\wss{What are your team's expectations regarding meeting attendance (being on
time, leaving early, missing meetings, etc.)?}

\subsubsection*{Acceptable Excuse}

\wss{What constitutes an acceptable excuse for missing a meeting or a deadline?
What types of excuses will not be considered acceptable?}

\subsubsection*{In Case of Emergency}

\wss{What process will team members follow if they have an emergency and cannot
attend a team meeting or complete their individual work promised for a team
deliverable?}

\subsection*{Accountability and Teamwork}

\subsubsection*{Quality} 

\wss{What are your team's expectations regarding the quality
of team members' preparation for team meetings and the quality of the
deliverables that members bring to the team?}

\subsubsection*{Attitude}

\wss{What are your team's expectations regarding team members' ideas,
interactions with the team, cooperation, attitudes, and anything else regarding
team member contributions?  Do you want to introduce a code of conduct?  Do you
want a conflict resolution plan?  Can adopt existing codes of conduct.}

\subsubsection*{Stay on Track}

\wss{What methods will be used to keep the team on track? How will your team
ensure that members contribute as expected to the team and that the team
performs as expected? How will your team reward members who do well and manage
members whose performance is below expectations?  What are the consequences for
someone not contributing their fair share?}

\wss{You may wish to use the project management metrics collected for the TA and
instructor for this.}

\wss{You can set target metrics for attendance, commits, etc.  What are the
consequences if someone doesn't hit their targets?  Do they need to bring the
coffee to the next team meeting?  Does the team need to make an appointment with
their TA, or the instructor?  Are there incentives for reaching targets early?}

\subsubsection*{Team Building}

\wss{How will you build team cohesion (fun time, group rituals, etc.)? }

\subsubsection*{Decision Making} 

\wss{How will you make decisions in your group? Consensus?  Vote? How will you
handle disagreements? }

\end{document}