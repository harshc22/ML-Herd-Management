\documentclass{article}

\usepackage{booktabs}
\usepackage{tabularx}
\usepackage{enumitem}

\title{Development Plan\\\progname}

\author{\authname}

\date{}

%% Comments

\usepackage{color}

\newif\ifcomments\commentstrue %displays comments
%\newif\ifcomments\commentsfalse %so that comments do not display

\ifcomments
\newcommand{\authornote}[3]{\textcolor{#1}{[#3 ---#2]}}
\newcommand{\todo}[1]{\textcolor{red}{[TODO: #1]}}
\else
\newcommand{\authornote}[3]{}
\newcommand{\todo}[1]{}
\fi

\newcommand{\wss}[1]{\authornote{blue}{SS}{#1}} 
\newcommand{\plt}[1]{\authornote{magenta}{TPLT}{#1}} %For explanation of the template
\newcommand{\an}[1]{\authornote{cyan}{Author}{#1}}

%% Common Parts

\newcommand{\progname}{ProgName} % PUT YOUR PROGRAM NAME HERE
\newcommand{\authname}{Team \#, Team Name
\\ Student 1 name
\\ Student 2 name
\\ Student 3 name
\\ Student 4 name} % AUTHOR NAMES                  

\usepackage{hyperref}
    \hypersetup{colorlinks=true, linkcolor=blue, citecolor=blue, filecolor=blue,
                urlcolor=blue, unicode=false}
    \urlstyle{same}
                                


\begin{document}

\maketitle

\begin{table}[hp]
\caption{Revision History} \label{TblRevisionHistory}
\begin{tabularx}{\textwidth}{llX}
\toprule
\textbf{Date} & \textbf{Developer(s)} & \textbf{Change}\\
\midrule
September 22 2024 & Martin Ivanov & Added team member roles and POC demonstration plan\\
September 22 2024 & Harshpreet Chinjer & Added meeting plan, Expected technology and Coding standard\\
September 23 2024 & Martin Ivanov & Added description for the Copyright License section\\
September 24 2024 & Martin Ivanov & Added Accountability and Teamwork goals to the Appendix\\
September 24 2024 & Harshpreet Chinjer & Added External goals and Attendance expectations\\
... & ... & ...\\
\bottomrule
\end{tabularx}
\end{table}

\newpage{}

\wss{Put your introductory blurb here.  Often the blurb is a brief roadmap of
what is contained in the report.}

\wss{Additional information on the development plan can be found in the
\href{https://gitlab.cas.mcmaster.ca/courses/capstone/-/blob/main/Lectures/L02b_POCAndDevPlan/POCAndDevPlan.pdf?ref_type=heads}
{lecture slides}.}

\section{Confidential Information?}

\wss{State whether your project has confidential information from industry, or
not.  If there is confidential information, point to the agreement you have in
place.}

\wss{For most teams this section will just state that there is no confidential
information to protect.}
\section{IP to Protect}

\wss{State whether there is IP to protect.  If there is, point to the agreement.
All students who are working on a project that requires an IP agreement are also
required to sign the ``Intellectual Property Guide Acknowledgement.''}

\section{Copyright License}
Having discussed with the industry supervisor, the project will be licensed under a proprietary license to protect
 the intellectual property and ensure that the project remains confidential. The proprietary
  license will allow the industry partner to use the software for their internal purposes which will allow them to have a competitive edge over the competing firms in the cattle management software industry.
  The proprietary license will also prevent the software from being open-sourced or made publicly available. 
  Currently, the team is awaiting the license agreement from the industry supervisor, which will be included in the project repository once received. This document will be updated to reflect that the license was received.

\section{Team Meeting Plan}

\begin{enumerate}[label=\textbf{\arabic*}]
    \item \textbf{Frequency of Meetings}
    \begin{itemize}
        \item \textbf{Team Meetings}: The team will meet virtually once a week at discussed time, with in-person meetings if required
        \item \textbf{Industry Supervisor Meetings}: Meetings with the industry supervisor will occur virtually once a week, with additional meetings or cancellations based on the supervisor's availability and project progress.
        \item \textbf{Additional Meetings}: The team or supervisor may call extra meetings as needed, especially during critical project milestones.
    \end{itemize}
    
    \item \textbf{Meeting Location}
    \begin{itemize}
        \item \textbf{Virtual Meetings}: Conducted via platforms Discord or Microsoft Teams.
        \item \textbf{In-Person Meetings}: Arranged at a mutually agreed location like Mills Memorial Library, H.G. Thode Library and ITB Hallway Space 
    \end{itemize}

    \item \textbf{Meeting Structure}
    \begin{itemize}
    \item \textbf{Chair Rotation}: Each team member will take turns chairing the meetings. This includes leading discussions, ensuring the agenda is followed, and managing time.
    \item \textbf{Agenda}: The chair of the upcoming meeting is responsible for preparing the agenda and circulating it to all members at least \textbf{24 hours prior} to the meeting. The agenda will include:
    \begin{itemize}
        \item Project updates
        \item Task allocation
        \item Deadlines review
        \item Challenges and risks
        \item Next steps
        \item Supervisor queries (for meetings with the industry advisor)
    \end{itemize}
    \end{itemize}

    \item \textbf{Meeting Protocol}
    \begin{itemize}
        \item \textbf{Attendance}: All members are expected to attend unless unable, in which case they should inform the members ahead of time.
        \item \textbf{Minutes}: One team member will be assigned to take minutes, including key decisions, action items, and next steps. Minutes will be shared within \textbf{24 hours} after the meeting on GitHub.
    \end{itemize}

\end{enumerate}


\section{Team Communication Plan}

\wss{Issues on GitHub should be part of your communication plan.}

\section{Team Member Roles}

\wss{You should identify the types of roles you anticipate, like notetaker,
leader, meeting chair, reviewer.  Assigning specific people to those roles is
not necessary at this stage.  In a student team the role of the individuals will
likely change throughout the year.}

\subsection*{Liaison} Responsible for communicating with the industry supervisor, ensuring that the project is on track, and addressing any concerns or questions from the supervisor.   
\subsection*{Chair} Responsible for leading meetings, ensuring the agenda is followed, and managing time. Team members will take turns being the chair for each meeting.
\subsection*{Meeting Minutes} Responsible for taking minutes during meetings, including key decisions, action items, and next steps. Meeting minutes will be taken by a single member at a time, and the member taking minutes will rotate every week.
\subsection*{Machine Learning Lead} Responsible for leading the machine learning component of the project, including researching machine learning models, and ensuring that the machine learning component meets the requirements.
\subsection*{Developer} Responsible for developing the project, including coding, testing, and documentation. Developers must ensure that the code produced for the project is of high quality and is well-documented.
\subsection*{Tester} Responsible for testing the project, including unit testing, integration testing, and system testing. This role will be responsible for ensuring that the project meets the requirements and is free of defects.
\subsection*{Path to Changing Roles} Team members can change roles based on their interest, availability, and skill set. The team will discuss and decide on role changes as needed. Furthermore, any team member may be asked to help out anywhere else in the system if the need arises. For this reason, it is important that all team members stay up to date with context for what is happening with different parts of the team. Team members may fulfill more than one role at once.

\section{Workflow Plan}

\begin{itemize}
	\item How will you be using git, including branches, pull request, etc.?
	\item How will you be managing issues, including template issues, issue
	classification, etc.?
  \item Use of CI/CD
\end{itemize}

\section{Project Decomposition and Scheduling}

\begin{itemize}
  \item How will you be using GitHub projects?
  \item Include a link to your GitHub project
\end{itemize}

\wss{How will the project be scheduled?  This is the big picture schedule, not
details. You will need to reproduce information that is in the course outline
for deadlines.}

\section{Proof of Concept Demonstration Plan}

A few potential challenges that may arise during the course of this project include the availability and quality of historical dairy farm data. While we will be provided with some datasets related to animal health, breeding success, and productivity, these may suffer from inconsistencies, missing values, or lack of relevant features. Our strategy to overcome this involves performing thorough data preprocessing steps to ensure the dataset is suitable for building a reliable model. Additionally, the complexity of predicting outcomes like breeding success or the likelihood of an animal leaving the herd poses a challenge in terms of model accuracy. To address this, we plan to conduct some experiments using various machine learning algorithms to identify the most accurate and robust model for these predictions.

For our POC demonstration in November, we will build a prototype model using a limited dataset that predicts breeding success rates. The POC will include a data pipeline that handles preprocessing steps, a prediction model that outputs breeding success probabilities, and means for displaying the predictions via a dashboard or command line output. This will allow us to demonstrate the viability of the system and our ability to overcome challenges related to data quality and prediction accuracy. In the final product, the model will be scaled to handle more complex data and generate additional insights such as health risks or productivity forecasts, but if the POC can show accurate predictions for a small subset of outcomes, it serves as proof that the system will work for more complex scenarios.

\section{Expected Technology}
\textbf{Initial Implementation Plan}

\begin{itemize}
    \item \textbf{Programming language}: 
    The project will primarily use \textbf{Python} due to its strong support for machine learning and data science tasks. For web development tasks, the project will use \textbf{React}, a \textbf{JavaScript} library. 
    \item \textbf{Libraries}: 
    Expected key libraries include \textbf{Pandas} and \textbf{Numpy} for data manipulation, \textbf{Scikit-learn}, \textbf{TensorFlow}, or \textbf{PyTorch} for building the machine learning model, and \textbf{Matplotlib} or \textbf{Seaborn} for data visualization.
    \item \textbf{AI Model}: 
    The project will create a \textbf{custom AI model} specifically tailored to dairy farming data, focusing on predicting outcomes such as breeding success rates and herd attrition.
    \item \textbf{Linter}: 
    \textbf{Flake8} or \textbf{Black} will be utilized to maintain code quality and ensure adherence to Python's PEP 8 standards.
    \item \textbf{Unit testing framework}: 
    \textbf{PyTest} will be used to implement unit tests, focusing on validating data pipelines, model training, and prediction accuracy.
    \item \textbf{Continuous Integration (CI)}: 
    \textbf{GitHub Actions} will be used to automate tests and ensure code quality in a continuous integration pipeline.
    \item \textbf{Version Control and Project Management}: 
    \textbf{Git} will be used for version control and \textbf{GitHub} for repository management and collaboration throughout the project.
\end{itemize}

\section{Coding Standard}

The project will follow the \textbf{PEP 8} coding standard for Python, which is the official style guide for Python code. PEP 8 outlines guidelines and best practices for writing clean, readable, and maintainable Python code. Key recommendations include:

\begin{itemize}
    \item Using 4 spaces per indentation level.
    \item Limiting line length to 79 characters.
    \item Proper use of blank lines to separate functions and classes.
    \item Consistent use of lower\_case\_with\_underscores for variable and function names.
    \item Avoiding extraneous whitespace.
    \item Using meaningful comments to explain the purpose of code blocks.
    \item Keeping code simple and readable, and adhering to Python’s philosophy of clarity.
    
\end{itemize}

For a full reference to PEP 8 guidelines, visit the official document: \href{https://peps.python.org/pep-0008/}{PEP 8 Style Guide}.


\newpage{}

\section*{Appendix --- Reflection}

\wss{Not required for CAS 741}

The purpose of reflection questions is to give you a chance to assess your own
learning and that of your group as a whole, and to find ways to improve in the
future. Reflection is an important part of the learning process.  Reflection is
also an essential component of a successful software development process.  

Reflections are most interesting and useful when they're honest, even if the
stories they tell are imperfect. You will be marked based on your depth of
thought and analysis, and not based on the content of the reflections
themselves. Thus, for full marks we encourage you to answer openly and honestly
and to avoid simply writing ``what you think the evaluator wants to hear.''

Please answer the following questions.  Some questions can be answered on the
team level, but where appropriate, each team member should write their own
response:


\begin{enumerate}
    \item Why is it important to create a development plan prior to starting the
    project?
    \item In your opinion, what are the advantages and disadvantages of using
    CI/CD?
    \item What disagreements did your group have in this deliverable, if any,
    and how did you resolve them?
\end{enumerate}

\newpage{}

\section*{Appendix --- Team Charter}

\subsection*{External Goals}

\begin{itemize}
    \item \textbf{Capstone EXPO Success}: The team aims to present the project at the Capstone EXPO and potentially win a prize by showcasing the innovative use of machine learning in the agricultural domain.
    
    \item \textbf{Portfolio and Interview Value}: Each member of the team intends to use this project as a strong portfolio piece, demonstrating their technical skills in machine learning, data science, and web development. The project will also provide valuable content to discuss in job interviews, highlighting problem-solving capabilities and teamwork.
    
    \item \textbf{High Academic Achievement}: The team is working toward achieving an A+ grade on this project, ensuring that all components meet the highest academic standards.
    
    \item \textbf{Networking and Industry Exposure}: Through regular meetings with the industry supervisor, the team hopes to build professional connections that could lead to future opportunities or collaborations within the industry.
    
    \item \textbf{Learning and Skill Enhancement}: Each member of the team is committed to learning new technologies and improving their skill set, particularly in machine learning, AI, and frontend development, as part of the project's process.
\end{itemize}


\subsection*{Attendance}

\subsubsection*{Expectations}

\begin{itemize}
    \item Team members are expected to attend all scheduled meetings, whether virtual or in-person, and be on time.
    \item Leaving early or missing a meeting without prior notice should be avoided unless absolutely necessary.
    \item Consistent attendance and active participation are critical to the success of the project, and all members are responsible for keeping up with the team's progress.
\end{itemize}

\subsubsection*{Acceptable/Unacceptable Excuse}

\begin{itemize}
    \item \textbf{Acceptable Excuses}: 
    Valid reasons include illness, family emergencies, or unavoidable academic or professional obligations (e.g., exams, presentations). In such cases, the member must notify the team as soon as possible.
    \item \textbf{Unacceptable Excuses}: 
    Personal convenience, poor time management, or non-urgent conflicts will not be considered acceptable reasons for missing a meeting or failing to meet a deadline.
\end{itemize}

\subsubsection*{In Case of Emergency}

\begin{itemize}
    \item The team member must immediately inform the rest of the team through Discord or Microsoft Teams.
    \item The team will determine how to redistribute the missing member's responsibilities for that week or project milestone to ensure no delays in the overall timeline.
    \item The team will aim to support the member in catching up as soon as possible once the emergency is resolved.
\end{itemize}

\subsection*{Accountability and Teamwork}

\subsubsection*{Quality} 

Our team expects all members to come to meetings well-prepared, having reviewed any 
relevant materials and completed their assigned tasks. Deliverables should be of high 
quality, adhering to the agreed-upon standards and guidelines. If a member is unable 
to meet these expectations, they should communicate this to the team in advance and 
seek assistance if needed. The code submitted by team members will be reviewed by 
other team members to ensure quality and consistency via comments on GitHub Pull Requests.

\subsubsection*{Attitude}

Our team expects all members to maintain a positive and respectful attitude towards each other. 
This includes being open to different ideas, providing feedback, and  
listening during discussions.

\textbf{Code of Conduct}:
\begin{itemize}
    \item Treat all team members with respect and courtesy.
    \item Be receptive to feedback and willing to make improvements.
    \item Maintain professionalism in all interactions, both within the team and with external stakeholders.
\end{itemize}

\subsubsection*{Stay on Track}

To ensure that the team stays on track, we will implement the following methods:

\begin{itemize}
    \item \textbf{Regular Check-ins}: We will use our Discord channel to discuss progress, address any issues, and adjust plans as necessary.
    \item \textbf{Task Tracking}: Use of GitHub Projects to track tasks and deadlines. Each task will be assigned to a team member, and progress will be monitored.
    \item \textbf{Performance Metrics}: Metrics such as the number of commits, issues resolved, and pull requests reviewed will be tracked to ensure active participation.
    \item \textbf{Peer Reviews}: Code and deliverables will be peer-reviewed to maintain quality and provide constructive feedback.
    \item \textbf{Consequences}: Members who do not contribute their fair share will be given a warning. Continued lack of contribution will result in a meeting with the TA or instructor to discuss further actions.
\end{itemize}

\subsubsection*{Team Building}

To build team cohesion, we plan to communicate primarily via Discord for regular updates, discussions, 
and casual interactions. Additionally, we will occasionally work together on campus, which will provide 
opportunities for face-to-face collaboration. These in-person sessions will help strengthen our teamwork 
and ensure that we stay aligned with our project goals.

\subsubsection*{Decision Making} 

We plan to use consensus for decision-making. However, if there is a major disagreement, we will resolve it by voting. Each team member will have an equal vote, and the majority decision will be implemented. In case of a tie, the team will discuss further to reach a consensus or consult the industry supervisor for guidance.

\end{document}