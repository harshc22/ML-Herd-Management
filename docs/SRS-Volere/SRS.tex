% THIS DOCUMENT IS FOLLOWS THE VOLERE TEMPLATE BY Suzanne Robertson and James Robertson
% ONLY THE SECTION HEADINGS ARE PROVIDED
%
% Initial draft from https://github.com/Dieblich/volere
%
% Risks are removed because they are covered by the Hazard Analysis
\documentclass[12pt]{article}

\usepackage{booktabs}
\usepackage{tabularx}
\usepackage{hyperref}
\hypersetup{
    bookmarks=true,         % show bookmarks bar?
      colorlinks=true,      % false: boxed links; true: colored links
    linkcolor=red,          % color of internal links (change box color with linkbordercolor)
    citecolor=green,        % color of links to bibliography
    filecolor=magenta,      % color of file links
    urlcolor=cyan           % color of external links
}

\newcommand{\lips}{\textit{Insert your content here.}}

%% Comments

\usepackage{color}

\newif\ifcomments\commentstrue %displays comments
%\newif\ifcomments\commentsfalse %so that comments do not display

\ifcomments
\newcommand{\authornote}[3]{\textcolor{#1}{[#3 ---#2]}}
\newcommand{\todo}[1]{\textcolor{red}{[TODO: #1]}}
\else
\newcommand{\authornote}[3]{}
\newcommand{\todo}[1]{}
\fi

\newcommand{\wss}[1]{\authornote{blue}{SS}{#1}} 
\newcommand{\plt}[1]{\authornote{magenta}{TPLT}{#1}} %For explanation of the template
\newcommand{\an}[1]{\authornote{cyan}{Author}{#1}}

%% Common Parts

\newcommand{\progname}{ProgName} % PUT YOUR PROGRAM NAME HERE
\newcommand{\authname}{Team \#, Team Name
\\ Student 1 name
\\ Student 2 name
\\ Student 3 name
\\ Student 4 name} % AUTHOR NAMES                  

\usepackage{hyperref}
    \hypersetup{colorlinks=true, linkcolor=blue, citecolor=blue, filecolor=blue,
                urlcolor=blue, unicode=false}
    \urlstyle{same}
                                


\begin{document}

\title{Software Requirements Specification for \progname: subtitle describing software} 
\author{\authname}
\date{\today}
	
\maketitle

~\newpage

\pagenumbering{roman}

\tableofcontents

~\newpage

\section*{Revision History}

\begin{tabularx}{\textwidth}{p{3cm}p{2cm}X}
\toprule {\textbf{Date}} & {\textbf{Version}} & {\textbf{Notes}}\\
\midrule
Date 1 & 1.0 & Notes\\
Date 2 & 1.1 & Notes\\
\bottomrule
\end{tabularx}

~\\

~\newpage
\section{Purpose of the Project}
\subsection{User Business}
\lips
\subsection{Goals of the Project}
\lips
\section{Stakeholders}
\subsection{Client}
\lips
\subsection{Customer}
\lips
\subsection{Other Stakeholders}
\lips
\subsection{Hands-On Users of the Project}
\lips
\subsection{Personas}
\lips
\subsection{Priorities Assigned to Users}
\lips
\subsection{User Participation}
\lips
\subsection{Maintenance Users and Service Technicians}
\lips

\section{Mandated Constraints}
\subsection{Solution Constraints}
\lips
\subsection{Implementation Environment of the Current System}
\lips
\subsection{Partner or Collaborative Applications}
\lips
\subsection{Off-the-Shelf Software}
\lips
\subsection{Anticipated Workplace Environment}
\lips
\subsection{Schedule Constraints}
\lips
\subsection{Budget Constraints}
\lips
\subsection{Enterprise Constraints}
\lips

\section{Naming Conventions and Terminology}
\subsection{Glossary of All Terms, Including Acronyms, Used by Stakeholders
involved in the Project}
\lips

\section{Relevant Facts And Assumptions}
\subsection{Relevant Facts}
\lips
\subsection{Business Rules}
\lips
\subsection{Assumptions}
\lips

\section{The Scope of the Work}
\subsection{The Current Situation}
\lips
\subsection{The Context of the Work}
\lips
\subsection{Work Partitioning}
\lips
\subsection{Specifying a Business Use Case (BUC)}
\lips

\section{Business Data Model and Data Dictionary}
\subsection{Business Data Model}
\lips
\subsection{Data Dictionary}
\lips

\section{The Scope of the Product}
\subsection{Product Boundary}
\lips
\subsection{Product Use Case Table}
\lips
\subsection{Individual Product Use Cases (PUC's)}
\lips

\section{Functional Requirements}
\subsection{Functional Requirements}
\lips

\section{Look and Feel Requirements}
\subsection{Appearance Requirements}
\lips
\subsection{Style Requirements}
\lips

\section{Usability and Humanity Requirements}
\subsection{Ease of Use Requirements}
\lips
\subsection{Personalization and Internationalization Requirements}
\lips
\subsection{Learning Requirements}
\lips
\subsection{Understandability and Politeness Requirements}
\lips
\subsection{Accessibility Requirements}
\lips

\section{Performance Requirements}
\subsection{Speed and Latency Requirements}
\lips
\subsection{Safety-Critical Requirements}
\lips
\subsection{Precision or Accuracy Requirements}
\lips
\subsection{Robustness or Fault-Tolerance Requirements}
\lips
\subsection{Capacity Requirements}
\lips
\subsection{Scalability or Extensibility Requirements}
\lips
\subsection{Longevity Requirements}
\lips

\section{Operational and Environmental Requirements}
\subsection{Expected Physical Environment}
\lips
\subsection{Wider Environment Requirements}
\lips
\subsection{Requirements for Interfacing with Adjacent Systems}
\lips
\subsection{Productization Requirements}
\lips
\subsection{Release Requirements}
\lips

\section{Maintainability and Support Requirements}
\subsection{Maintenance Requirements}
\lips
\subsection{Supportability Requirements}
\lips
\subsection{Adaptability Requirements}
\lips

\section{Security Requirements}
\subsection{Access Requirements}
\lips
\subsection{Integrity Requirements}
\lips
\subsection{Privacy Requirements}
\lips
\subsection{Audit Requirements}
\lips
\subsection{Immunity Requirements}
\lips

\section{Cultural Requirements}
\subsection{Cultural Requirements}
\begin{itemize}
	\item The primary language for the product will be English, tailored
	      specifically to Canadian dairy farmers. 
	\item All data and measurements will follow Canadian standards, including the
	      use of liters for milk production, kilograms for weight, and hectares
	      for land area Other relevant units such as Celsius for temperature and
	      metric tons for larger quantities may also be used
\end{itemize}

\section{Compliance Requirements}
\subsection{Legal Requirements}
\begin{itemize}
	\item The project must comply with the
	      \href{https://www.nfacc.ca/codes-of-practice/dairy-cattle}{Code of
	      Practice for the Care and Handling of Dairy Cattle}, which is a
	      government-regulated standard in Canada. This code outlines mandatory
	      guidelines for the ethical treatment, health, and welfare of dairy
	      cattle. Any management recommendations or actions suggested by the
	      machine learning model will align with these regulations to ensure
	      ethical practices in dairy farming.
	        
	\item The project must comply with
	      \href{https://laws-lois.justice.gc.ca/pdf/p-8.6.pdf}{PIPEDA} (Personal
	      Information Protection and Electronic Documents Act) for any personal
	      information related to dairy farmers or other individuals involved. This
	      includes the handling of contact details, financial information, and
	      other personally identifiable data.
\end{itemize}
\subsection{Standards Compliance Requirements}
\begin{itemize}
	\item There are no specific standards for collecting dairy farming data in
	      this project. All relevant aspects of data collection and handling are
	      already covered under Legal Requirements, specifically in compliance
	      with PIPEDA for managing sensitive information about dairy farmers, and
	      the Code of Practice for the Care and Handling of Dairy Cattle for
	      ensuring the welfare of the animals.
	\item For coding standards, the project will adhere to PEP8 to ensure
	      consistent and readable Python code. More information on PEP8 can be
	      found \href{https://peps.python.org/pep-0008/}{here}.
\end{itemize}

\section{Open Issues}
\begin{itemize}
	\item \textbf{Data Availability and Quality:} The accuracy of predictions will
	      heavily depend on the quality and completeness of the data obtained from
	      CATTLEytics and Lactanet. Inconsistent or missing data might affect the
	      performance of the model.
	      
	\item \textbf{Model Accuracy:} The machine learning model may need to be
	      fine-tuned multiple times to achieve high accuracy in predicting cow
	      traits. This requires testing with diverse datasets to ensure the model
	      generalizes well.
	      
	\item \textbf{User Interface Usability:} The graphical representation of the
	      family tree and predicted traits needs to be intuitive and user-friendly
	      for farmers with varying levels of technical skill. Determining the best
	      design and ensuring it meets users' needs could take time.
	      
	\item \textbf{Integration with CATTLEytics:} Seamlessly integrating the tool
	      into the existing CATTLEytics system without causing disruptions or
	      requiring major system changes could be technically challenging.
	      
	\item \textbf{Regulatory Compliance:} Ensuring that the predictions and
	      recommendations made by the model comply with Canadian regulations for
	      dairy farming (Code of Practice for the Care and Handling of Dairy
	      Cattle) will require thorough review and potential adjustments during
	      development.
	      
	\item \textbf{Model Interpretability:} Farmers may need clear explanations for
	      how predictions are made to trust and use the tool effectively. Ensuring
	      the model’s predictions are explainable is an open issue.
	      
	\item \textbf{Performance Considerations:} The tool needs to be efficient and
	      scalable, handling large amounts of data without significant lag or
	      performance issues, especially as it gets adopted by multiple farms.
\end{itemize}

\section{Off-the-Shelf Solutions}
\subsection{Ready-Made Products}
\begin{itemize}
	\item There are no fully ready-made products that address the predictive
	      capabilities being developed in this project. While tools like Lactanet
	      provide dairy farm data, they do not offer predictive models based on
	      genetic and health data. Lactanet data will be used primarily for
	      training the custom machine learning model.
\end{itemize}
\subsection{Reusable Components}
\begin{itemize}
	\item Machine learning libraries, such as PyTorch or TensorFlow, will be
	      utilized to develop the custom AI model for cow trait prediction.
	      Additionally, front-end libraries such as D3.js or React Tree
	      Visualization libraries could be considered for visualizing the
	      family-tree diagrams.
\end{itemize}
\subsection{Products That Can Be Copied}
\begin{itemize}
	\item There are no existing products to be copied for this project. However,
	      open-source family-tree visualization tools might serve as inspiration
	      for the graphical aspects of the project.
\end{itemize}

\section{New Problems}
\subsection{Effects on the Current Environment}
\begin{itemize}
	\item Introducing this system could change how farmers currently select sires
	      or evaluate herd performance. Some may resist adopting new technology
	      due to unfamiliarity.
\end{itemize}
\subsection{Effects on the Installed Systems}
\begin{itemize}
	\item The project will be integrated into the existing Cattleytics software,
	      which is already used to manage dairy farms. The machine learning tool
	      will act as an additional module within Cattleytics, allowing farmers to
	      visualize the family tree of cows and predict future traits based on
	      genetic data. Seamless integration with the current system will be
	      prioritized to ensure smooth adoption and ease of use.
\end{itemize}
\subsection{Potential User Problems}
\begin{itemize}
	\item Users may face difficulties interpreting complex AI model outputs, so
	      ensuring the tool’s recommendations are easy to understand is key.
\end{itemize}
\subsection{Limitations in the Anticipated Implementation Environment That May
Inhibit the New Product}
\begin{itemize}
	\item The tool will need to function effectively on standard farm computing
	      systems, which may have limited processing power or internet
	      connectivity.
\end{itemize}
\subsection{Follow-Up Problems}
\begin{itemize}
	\item Continuous updates may be needed to improve the model based on feedback
	      from farmers. Future updates may also need to address changes in farming
	      practices
\end{itemize}

\section{Tasks}
\subsection{Project Planning}
\lips
\subsection{Planning of the Development Phases}
\lips

\section{Migration to the New Product}
\subsection{Requirements for Migration to the New Product}
\lips
\subsection{Data That Has to be Modified or Translated for the New System}
\lips

\section{Costs}
\lips
\section{User Documentation and Training}
\subsection{User Documentation Requirements}
\lips
\subsection{Training Requirements}
\lips

\section{Waiting Room}
\lips

\section{Ideas for Solution}
\lips

\newpage{}
\section*{Appendix --- Reflection}

The information in this section will be used to evaluate the team members on the
graduate attribute of Lifelong Learning.  Please answer the following questions:

\begin{enumerate}
  \item What knowledge and skills will the team collectively need to acquire to
  successfully complete this capstone project?  Examples of possible knowledge
  to acquire include domain specific knowledge from the domain of your
  application, or software engineering knowledge, mechatronics knowledge or
  computer science knowledge.  Skills may be related to technology, or writing,
  or presentation, or team management, etc.  You should look to identify at
  least one item for each team member.
  \item For each of the knowledge areas and skills identified in the previous
  question, what are at least two approaches to acquiring the knowledge or
  mastering the skill?  Of the identified approaches, which will each team
  member pursue, and why did they make this choice?
\end{enumerate}

\end{document}