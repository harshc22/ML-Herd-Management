% THIS DOCUMENT IS FOLLOWS THE VOLERE TEMPLATE BY Suzanne Robertson and James Robertson
% ONLY THE SECTION HEADINGS ARE PROVIDED
%
% Initial draft from https://github.com/Dieblich/volere
%
% Risks are removed because they are covered by the Hazard Analysis
\documentclass[12pt]{article}

\usepackage{booktabs}
\usepackage{tabularx}
\usepackage{hyperref}
\hypersetup{
    bookmarks=true,         % show bookmarks bar?
      colorlinks=true,      % false: boxed links; true: colored links
    linkcolor=red,          % color of internal links (change box color with linkbordercolor)
    citecolor=green,        % color of links to bibliography
    filecolor=magenta,      % color of file links
    urlcolor=cyan           % color of external links
}

\newcommand{\lips}{\textit{Insert your content here.}}

%% Comments

\usepackage{color}

\newif\ifcomments\commentstrue %displays comments
%\newif\ifcomments\commentsfalse %so that comments do not display

\ifcomments
\newcommand{\authornote}[3]{\textcolor{#1}{[#3 ---#2]}}
\newcommand{\todo}[1]{\textcolor{red}{[TODO: #1]}}
\else
\newcommand{\authornote}[3]{}
\newcommand{\todo}[1]{}
\fi

\newcommand{\wss}[1]{\authornote{blue}{SS}{#1}} 
\newcommand{\plt}[1]{\authornote{magenta}{TPLT}{#1}} %For explanation of the template
\newcommand{\an}[1]{\authornote{cyan}{Author}{#1}}

%% Common Parts

\newcommand{\progname}{ProgName} % PUT YOUR PROGRAM NAME HERE
\newcommand{\authname}{Team \#, Team Name
\\ Student 1 name
\\ Student 2 name
\\ Student 3 name
\\ Student 4 name} % AUTHOR NAMES                  

\usepackage{hyperref}
    \hypersetup{colorlinks=true, linkcolor=blue, citecolor=blue, filecolor=blue,
                urlcolor=blue, unicode=false}
    \urlstyle{same}
                                


\begin{document}

\title{Software Requirements Specification for \progname: subtitle describing software} 
\author{\authname}
\date{\today}
	
\maketitle

~\newpage

\pagenumbering{roman}

\tableofcontents

~\newpage

\section*{Revision History}

\begin{tabularx}{\textwidth}{p{3cm}p{2cm}X}
\toprule {\textbf{Date}} & {\textbf{Version}} & {\textbf{Notes}}\\
\midrule
October 6, 2024 & 1.0 & Initial SRS document\\
\bottomrule
\end{tabularx}

~\\

~\newpage
\section{Purpose of the Project}
\subsection{User Business}

Dairy farmers face challenges in selecting the right sires and dams for 
breeding to achieve desirable traits in their herds, such as high milk 
production or reduced susceptibility to diseases like lameness. Traditional 
breeding decisions are often based on trial and error or limited genetic 
information, leading to inefficiencies, lower productivity, and increased costs 
due to poor breeding outcomes. \\

The machine learning model developed in this project will allow farmers to make 
data-driven breeding decisions. By predicting the probability of specific 
traits (like lameness or milk yield) in offspring based on parental data, 
farmers can:
\begin{itemize}
    \item Reduce the risk of passing on undesirable traits.
    \item Optimize the genetic quality of the herd.
    \item Improve productivity by selecting animals that are likely to perform better in terms of milk yield, health, and longevity.
\end{itemize} 

The insights generated by the model will enable farmers to adopt a more 
proactive approach to herd management, minimizing risks and improving the 
efficiency of breeding programs.


\subsection{Goals of the Project}
\begin{itemize}
    \item \textbf{Predict Offspring Traits}:
    \begin{itemize}
        \item Develop a machine learning model that can predict the likelihood 
        of specific traits, such as lameness or milk production, in the 
        offspring of a given sire and dam.
        \item Ensure that the model provides accurate predictions (target 
        accuracy 80-90\%) based on available historical data.
    \end{itemize}
    
    \item \textbf{Seamless Integration into Existing Systems}:
    \begin{itemize}
        \item Ensure that the machine learning model integrates seamlessly with 
        CATTLEytics Inc.'s existing farm management platform without disrupting 
        the current workflows.
    \end{itemize}

    \item \textbf{Improve Breeding Decisions}:
    \begin{itemize}
        \item Provide actionable insights that allow farmers to make 
        better-informed breeding decisions.
    \end{itemize}
    
    \item \textbf{Enhance Herd Productivity}:
    \begin{itemize}
        \item By optimizing breeding decisions, the system aims to reduce 
        undesirable traits and improve herd productivity.
    \end{itemize}

    \item \textbf{User-Friendly Interface}:
    \begin{itemize}
        \item Ensure that the output from the machine learning model is clear 
        and easy to understand allowing for users with varying levels of 
        techinical expertise to easily use.
    \end{itemize}
\end{itemize}



\section{Stakeholders}

\subsection{Client}
The client for this project is CATTLEytics Inc., a company that provides farm 
management solutions to dairy farmers. CATTLEytics Inc. is responsible for 
overseeing the project development, providing data, and ensuring that the final 
product integrates seamlessly with their existing platform. The client will 
also ensure the proprietary license is respected and that intellectual property 
remains with the company.

\subsection{Customer}
The customers are dairy farmers who will directly use the machine learning 
model integrated into the CATTLEytics Inc. platform. These farmers are 
primarily interested in improving herd management through more informed 
breeding decisions.

\subsection{Other Stakeholders}
Other stakeholders include:

\begin{itemize}
    \item \textbf{Animal Health Experts}: These professionals may use the 
    system to advise farmers on breeding decisions by analyzing predictions 
    related to animal health and genetics.
    \item \textbf{Technology Partners}: Providers of data (e.g., Lactanet) or 
    cloud services that may contribute to the data and technical infrastructure 
    required for model development and deployment.
\end{itemize}

\subsection{Hands-On Users of the Project}
The primary hands-on users are dairy farmers who will interact with the system 
through the CATTLEytics Inc. platform. The system will provide them with 
actionable insights into breeding decisions based on the predicted traits of 
the offspring.

\subsection{Personas}
\begin{itemize}
    \item \textbf{Farmer}: Typically manages a large herd and is interested in 
    optimizing breeding decisions to improve overall herd productivity and 
    health.

\end{itemize}

\subsection{Priorities Assigned to Users}
Farmers, as primary users, are assigned the highest priority, as they will be 
the ones making decisions based on the model's predictions.Animal health 
experts and other stakeholders have a secondary priority as they offer support 
and advisory services based on the results generated by the system.

\subsection{User Participation}
Farmers will actively participate in providing feedback on the system's ease of 
use and accuracy of predictions. Their feedback will be crucial in refining the 
model and improving the interface. Additionally, CATTLEytics Inc. will ensure 
the model remains aligned with business objectives and customer needs.

\subsection{Maintenance Users and Service Technicians}
The CATTLEytics Inc. development team and service technicians will maintain the 
system. They will ensure tha the model continues to perfom as expected and 
continues smoothly integrate with the farm management techonolgoy. These users 
are responsible for addressing technical issues, deploying updates, and 
ensuring data security and privacy compliance.

\section{Mandated Constraints}

\subsection{Solution Constraints}
\begin{itemize}
    \item The solution must leverage ML techniques to predict offspring traits 
    based on parental data, ensuring that predictions are derived from 
    historical data sets relevant to dairy farming.
    \item The model must be able to provide predictions with a target accuracy 
    of 80-90\%, based on the training data available. It should also be robust 
    enough to handle missing data effectively. 
\end{itemize}

\subsection{Implementation Environment of the Current System}
\begin{itemize}
    \item The machine learning model will be integrated into CATTLEytics Inc.'s 
    existing farm management platform.
    \item The system should be deployable on standard hardware typically found 
    in farm management environments, ensuring compatibility with existing 
    software solutions used by users.
\end{itemize}

\subsection{Partner or Collaborative Applications}
\begin{itemize}
    \item The project may collaborate with external data providers, such as 
    Lactanet, to access relevant genetic and productivity data for training the 
    model.
\end{itemize}

\subsection{Off-the-Shelf Software}
\begin{itemize}
    \item There will likely be a need to use off-the-shelf software to build 
    and test the ML model. This may include and is not limited to Python's 
    TensorFlow and Pandas library annd some cloud database like AWS. 
\end{itemize}

\subsection{Anticipated Workplace Environment}
\begin{itemize}
    \item The primary users will be dairy farmers who operate in rural 
    environments, where access to high-speed internet may be limited.
\end{itemize}

\subsection{Schedule Constraints}
\begin{itemize}
    \item The project is expected to adhere to a defined timeline set by 
    CATTLEytics Inc., with major milestones including:
    \begin{itemize}
        \item Completion of the initial proof of concept by \textbf{November 
        22, 2024}.
        \item The final demonstration and documentation due by \textbf{April 2, 
        2025}.
    \end{itemize}
    \item Any delays in data acquisition or model development may affect the 
    project timeline.
\end{itemize}

\subsection{Budget Constraints}
\begin{itemize}
    \item The project must be completed within the budget allocated by 
    CATTLEytics Inc. This budget may limit the resources available for 
    development, testing, and deployment. These resources will likely be used 
    towards hardware, cloud storage, and commerical software. 
\end{itemize}

\subsection{Enterprise Constraints}
\begin{itemize}
    \item The developed system must comply with the enterprise policies of 
    CATTLEytics Inc., particularly regarding data handling and user access 
    control.
\end{itemize}


\section{Naming Conventions and Terminology}
\subsection{Glossary of All Terms, Including Acronyms, Used by Stakeholders
involved in the Project}

\begin{tabularx}{\textwidth}{|l|X|}
    \hline
    \textbf{Term} & \textbf{Definition} \\
    \hline
    ML (Machine Learning) & A branch of AI that enables systems to learn from 
    data and make predictions. \\
    \hline
    IoT (Internet of Things) & A system of interconnected devices that collect 
    and exchange data, potentially including sensors in farm equipment. \\
    \hline
    Sire & The male parent (father) in breeding. \\
    \hline
    Dam & The female parent (mother) in breeding. \\
    \hline
    Breeding Success Rate & The likelihood of a successful pairing between a 
    sire and dam to produce desired offspring traits. \\
    \hline
\end{tabularx}

\section{Relevant Facts and Assumptions}

\subsection{Relevant Facts}
\begin{itemize}
    \item The project is focused on developing a machine learning model 
    specifically for predicting offspring traits in dairy cattle based on 
    parental traits.
    \item Historical data, including health, breeding attempts, and 
    productivity records, will be utilized to train the machine learning model.
    \item The primary end-users of the system are dairy farmers, who will 
    access the model through CATTLEytics Inc.'s existing farm management 
    platform.
    \item The project will operate under a proprietary license, ensuring that 
    CATTLEytics Inc. retains ownership of the intellectual property.
\end{itemize}

\subsection{Business Rules}
\begin{itemize}
    \item \textbf{Data Integrity:} All input data used for training and 
    predictions must be accurate, complete, and obtained from verified sources 
    to ensure the reliability of the predictions made by the model.
    \item \textbf{User Access:} Only authorized users (i.e., registered dairy 
    farmers and CATTLEytics Inc. personnel) will have access to the system to 
    maintain data security and privacy.
    \item \textbf{Compliance:} The system must comply with relevant data 
    protection regulations, such as the General Data Protection Regulation 
    (GDPR) and Canadian privacy laws, ensuring that user data is handled 
    securely and ethically.
    \item \textbf{Feedback Mechanism:} A mechanism will be established to 
    collect feedback from users about the accuracy of predictions and the 
    usability of the system, which will inform future improvements and 
    iterations.
\end{itemize}

\subsection{Assumptions}
\begin{itemize}
    \item It is assumed that sufficient historical data related to herd health, 
    breeding, and productivity is available for model training and validation.
    \item The target user group (dairy farmers) is expected to have a basic 
    level of technological literacy to interact effectively with the system.
    \item It is assumed that the integration of the machine learning model into 
    CATTLEytics Inc.'s existing platform will proceed smoothly without 
    significant disruption to farmers' workflows.
    \item The project assumes that farmers will be willing to provide feedback 
    and participate in the evaluation of the model's performance to enhance its 
    accuracy and usability.
    \item The system is expected to rely primarily on historical data, with 
    limited real-time data collection due to potential connectivity issues in 
    rural areas.
\end{itemize}


\section{The Scope of the Work}
\subsection{The Current Situation}
The current state of dairy farming presents challenges in efficiently predicting
the health, productivity, and breeding outcomes of cattle. Farmers typically 
rely on historical records, but the analysis is done manually, often leading to 
reactive management. The existing systems do not utilize advanced technologies 
such as machine learning for predictive analytics. As a result, there is 
limited proactive management regarding milk production, breeding success, and 
herd longevity, which directly impacts farm profitability and sustainability.

The current solution environment lacks integration of large datasets from 
multiple sources, such as individual cow health records, breeding history, and 
environmental conditions, into a single system that can offer actionable 
predictions.

\subsection{The Context of the Work}
This project aims to develop a machine learning model that will leverage 
historical herd data to predict important traits such as milk yield, breeding 
success rates, and the likelihood of a cow leaving the herd. This model will be 
integrated into a farm management system, providing farmers with actionable 
insights. The goal is to move from reactive to proactive herd management.

The model will use data such as the health, breed, and genetic history of both 
the mother and father to predict traits in calves. The software will be 
developed as part of a partnership with CATTLEytics Inc., ensuring seamless 
integration into their existing platform used by dairy farmers.

\subsection{Work Partitioning}
The project will be divided into several key components:

\begin{enumerate}
    \item \textbf{Data Collection and Preprocessing:}
    \begin{itemize}
        \item Collection of historical data from existing systems, including 
        cow health records, breeding data, and productivity metrics.
        \item Cleaning and standardizing the data for input into the machine 
        learning model.
    \end{itemize}

    \item \textbf{Model Development:}
    \begin{itemize}
        \item Designing and implementing the machine learning model for trait 
        prediction (e.g., milk production, herd retention).
        \item Training and validating the model on historical datasets.
        \item Iterative testing and refinement.
    \end{itemize}

    \item \textbf{Integration:}
    \begin{itemize}
        \item Integrating the prediction model into the CATTLEytics Inc. farm 
        management system.
        \item Ensuring that outputs are presented in a user-friendly format 
        for farmers to make decisions.
    \end{itemize}

    \item \textbf{Testing and Validation:}
    \begin{itemize}
        \item Testing the software in real-world farm environments to validate 
        predictions and refine the user interface.
    \end{itemize}

    \item \textbf{Documentation and Training:}
    \begin{itemize}
        \item Providing clear documentation for users and training for farmers 
        to effectively use the system.
    \end{itemize}
\end{enumerate}

\subsection{Specifying a Business Use Case (BUC)}

\textbf{Title:} Predicting Cow Traits for Optimized Herd Management\\
\textbf{Primary Actor:} Dairy farmer using the CATTLEytics Inc. system.\\
\textbf{Precondition:} The farmer has access to a herd management system 
integrated with the prediction model. Historical data on breeding, milk 
production, and herd turnover are available.\\
\textbf{Trigger:} The farmer initiates the model to predict the outcomes of a 
planned breeding or evaluates the likelihood of an existing cow leaving the 
herd.\\
\textbf{Main Success Scenario:}
\begin{enumerate}
    \item The farmer selects cows for breeding and inputs the necessary data 
    (e.g., parent traits).
    \item The system processes the input and returns predictions for milk yield 
    and herd retention likelihood.
    \item Based on the model's predictions, the farmer makes informed decisions 
    on breeding strategies or management actions to prevent herd loss.
\end{enumerate}
\textbf{Postconditions:}
\begin{itemize}
    \item The farmer has actionable insights to improve herd productivity and 
    manage herd turnover proactively.
\end{itemize}

\section{Business Data Model and Data Dictionary}
\subsection{Business Data Model}
This section will be completed once the relevant data model details are 
available.
\subsection{Data Dictionary}
This section will be completed once the relevant data model details are 
available.


\section{The Scope of the Product}


\subsection{Product Boundary}
The product's primary function is to predict cow traits such as milk production,
breeding success, and herd retention based on parental data. The machine 
learning model will integrate into a platform like CATTLEytics Inc., and its 
boundaries will include:

\begin{itemize}
    \item \textbf{Included}: The product will use historical data to generate 
    predictive insights on milk production, breeding success rates, and herd 
    retention likelihood. Farmers will be able to input relevant data to 
    receive predictions.
    \item \textbf{Not Included}: Real-time data collection and analysis, 
    live health monitoring, or any complex integrations with external devices 
    (e.g., IoT sensors).
\end{itemize}

This product will focus solely on predictive analytics based on historical data 
and will not handle aspects like external raw data collection or advanced herd 
management automation beyond providing insights.

\subsection{Product Use Case Table}
This table outlines the core use cases currently identified for the product.

\begin{table}[h!]
\centering
\begin{tabularx}{\textwidth}{|p{2cm}|p{3cm}|p{3cm}|X|}
    \hline
    \textbf{ID} & \textbf{Title} & \textbf{Actor} & \textbf{Description} \\
    \hline
    PUC1 & Predict Breeding Success & Dairy Farmer & Farmers input breeding 
    data to get predictions on the likelihood of successful breeding. \\
    \hline
    PUC2 & Forecast Milk Production & Dairy Farmer & Farmers input cow data 
    to get a prediction of future milk production. \\
    \hline
    PUC3 & Predict Herd Retention Likelihood & Dairy Farmer & Farmers receive 
    predictions on whether cows are likely to stay in or leave the herd. \\
    \hline
\end{tabularx}
\end{table}


\subsection{Individual Product Use Cases (PUC's)}

\subsubsection{PUC1: Predict Breeding Success}
\begin{itemize}
    \item \textbf{Primary Actor}: Dairy Farmer
    \item \textbf{Preconditions}: Farmer has historical data about cow and 
    parental traits available for input.
    \item \textbf{Trigger}: The farmer initiates a request to predict 
    breeding success.
    \item \textbf{Main Success Scenario}:
    \begin{enumerate}
        \item The farmer enters relevant breeding data.
        \item The system processes the input using historical records.
        \item A prediction is generated on the likelihood of breeding success.
    \end{enumerate}
    \item \textbf{Postcondition}: The farmer gets an actionable prediction 
    to decide whether to proceed with the breeding.
\end{itemize}

\subsubsection{PUC2: Forecast Milk Production}
\begin{itemize}
    \item \textbf{Primary Actor}: Dairy Farmer
    \item \textbf{Preconditions}: Historical data for milk production and 
    parental traits is available for input.
    \item \textbf{Trigger}: The farmer requests a prediction for future milk 
    production.
    \item \textbf{Main Success Scenario}:
    \begin{enumerate}
        \item The farmer inputs the cow's data.
        \item The system processes the input data.
        \item A prediction on future milk production is generated.
    \end{enumerate}
    \item \textbf{Postcondition}: The farmer receives a prediction that helps 
    in planning milk yield expectations.
\end{itemize}

\subsubsection{PUC3: Predict Herd Retention Likelihood}
\begin{itemize}
    \item \textbf{Primary Actor}: Dairy Farmer
    \item \textbf{Preconditions}: Health and productivity data is available 
    for the cows in question.
    \item \textbf{Trigger}: The farmer requests predictions on herd retention 
    likelihood.
    \item \textbf{Main Success Scenario}:
    \begin{enumerate}
        \item The farmer selects a cow or group of cows for analysis.
        \item The system processes the available data.
        \item A prediction is generated on whether the cows are likely to stay 
        in or leave the herd.
    \end{enumerate}
    \item \textbf{Postcondition}: The farmer receives predictions to assist in 
    managing herd turnover.
\end{itemize}


\subsubsection{PUC1: Predict Breeding Success}
\begin{itemize}
    \item \textbf{Primary Actor}: Dairy Farmer
    \item \textbf{Preconditions}: Farmer has historical data about cow and 
    parental traits available for input.
    \item \textbf{Trigger}: The farmer initiates a request to predict 
    breeding success.
    \item \textbf{Main Success Scenario}:
    \begin{enumerate}
        \item The farmer enters relevant breeding data.
        \item The system processes the input using historical records.
        \item A prediction is generated on the likelihood of breeding success.
    \end{enumerate}
    \item \textbf{Postcondition}: The farmer gets an actionable prediction 
    to decide whether to proceed with the breeding.
\end{itemize}

\subsubsection{PUC2: Forecast Milk Production}
\begin{itemize}
    \item \textbf{Primary Actor}: Dairy Farmer
    \item \textbf{Preconditions}: Historical data for milk production and 
    parental traits is available for input.
    \item \textbf{Trigger}: The farmer requests a prediction for future milk 
    production.
    \item \textbf{Main Success Scenario}:
    \begin{enumerate}
        \item The farmer inputs the cow's data.
        \item The system processes the input data.
        \item A prediction on future milk production is generated.
    \end{enumerate}
    \item \textbf{Postcondition}: The farmer receives a prediction that helps 
    in planning milk yield expectations.
\end{itemize}

\subsubsection{PUC3: Predict Herd Retention Likelihood}
\begin{itemize}
    \item \textbf{Primary Actor}: Dairy Farmer
    \item \textbf{Preconditions}: Health and productivity data is available 
    for the cows in question.
    \item \textbf{Trigger}: The farmer requests predictions on herd retention 
    likelihood.
    \item \textbf{Main Success Scenario}:
    \begin{enumerate}
        \item The farmer selects a cow or group of cows for analysis.
        \item The system processes the available data.
        \item A prediction is generated on whether the cows are likely to stay 
        in or leave the herd.
    \end{enumerate}
    \item \textbf{Postcondition}: The farmer receives predictions to assist in 
    managing herd turnover.
\end{itemize}


\section{Functional Requirements}

\subsection{Functional Requirements}

\textbf{FR1: Predict Breeding Success}
\begin{itemize}
    \item \textbf{Description}: The system shall predict the likelihood of a 
    successful breeding event between two cows based on input data regarding 
    parental traits and historical breeding records.
    \item \textbf{Rationale}: This feature will help farmers make more informed 
    breeding decisions, improving breeding efficiency and reducing failures.
    \item \textbf{Fit Criterion}: The system will output a probability of 
    breeding success based on parental data, and this probability must be 
    verified by comparing predicted outcomes with actual breeding success over 
    time.
\end{itemize}
\textbf{FR2: Forecast Milk Production}
\begin{itemize}
    \item \textbf{Description}: The system shall forecast the milk production 
    of a cow based on historical milk yield and parental traits.
    \item \textbf{Rationale}: Accurate predictions of future milk yield will 
    enable farmers to better plan for production and make decisions on herd 
    management.
    \item \textbf{Fit Criterion}: The forecasted milk production must be within 
    10\% (to be determined) accuracy when compared to actual milk yield over a 
    specified period.
\end{itemize}
\textbf{FR3: Predict Herd Retention Likelihood}
\begin{itemize}
    \item \textbf{Description}: The system shall predict the likelihood of 
    cows leaving the herd based on health records, productivity, and other 
    historical data.
    \item \textbf{Rationale}: This feature will enable farmers to proactively 
    manage their herd, reducing unexpected departures and improving herd 
    stability.
    \item \textbf{Fit Criterion}: The system will provide a prediction score 
    (e.g., high, medium, low) for herd retention, which can be evaluated by 
    tracking actual herd retention over a six-month period.
\end{itemize}
\textbf{FR4: Data Input for Predictions}
\begin{itemize}
    \item \textbf{Description}: The system shall allow the farmer to input 
    relevant data, such as breeding records, milk production history, and 
    health records, into the prediction model.
    \item \textbf{Rationale}: To generate accurate predictions, the system 
    requires access to a range of historical data that can be inputted by the 
    user.
    \item \textbf{Fit Criterion}: The input form must successfully accept and 
    validate required fields for at least 95\% of user inputs, with clear error 
    handling for missing or incorrect data.
\end{itemize}
\textbf{FR5: Report Generation}
\begin{itemize}
    \item \textbf{Description}: The system shall generate a report summarizing 
    predictions for breeding success, milk production, and herd retention for 
    selected cows.
    \item \textbf{Rationale}: Farmers need a consolidated report that provides 
    actionable insights based on the predictions generated by the system.
    \item \textbf{Fit Criterion}: The system will generate reports that can be 
    exported to a PDF format and include all requested predictions in a 
    structured layout.
\end{itemize}
\textbf{FR6: User Access Control}
\begin{itemize}
    \item \textbf{Description}: The system shall provide secure login and 
    role-based access control, ensuring that only authorized users can access 
    or modify the prediction data.
    \item \textbf{Rationale}: Farm management data is sensitive and should only 
    be accessible by authorized personnel.
    \item \textbf{Fit Criterion}: The system must enforce unique login 
    credentials for each user and restrict access based on roles (e.g., farmer, 
    supervisor), with at least 99\% reliability in access control enforcement.
\end{itemize}
\textbf{FR7: Integration with Farm Management System}
\begin{itemize}
    \item \textbf{Description}: The system shall be designed to integrate with 
    existing farm management platforms, such as CATTLEytics Inc, allowing 
    seamless data exchange.
    \item \textbf{Rationale}: Integration with existing platforms will enable 
    the system to leverage historical data and provide predictions without 
    requiring manual data entry.
    \item \textbf{Fit Criterion}: The system should successfully exchange data 
    with the farm management platform 90\% of the time during testing, 
    without errors in data transmission.
\end{itemize}

\subsection{Formal Specification}

\textbf{Specification 1: Breeding Success Prediction}
\begin{itemize}
    \item \textbf{Description}: The system must be able to predict the 
    likelihood of breeding success between two cows based on historical data, 
    such as parental traits and previous breeding records.
    \item \textbf{Formal Specification}: \\
    Let \( X \) represent a breeding event. \\
    Let \( Y \) represent the set of all possible breeding events. \\
    Let \( P \) represent the predicted probability of success.
    
    \[
    \forall X \in Y : \textrm{Prediction}(X) \rightarrow P \in [0, 1]
    \]
    
    The system shall compute the probability \( P \) for each breeding event \( 
    X \).
\end{itemize}

\textbf{Specification 2: Milk Production Forecast}
\begin{itemize}
    \item \textbf{Description}: The system shall forecast future milk 
    production for a given cow based on historical data of both the cow and its 
    parents.
    \item \textbf{Formal Specification}: \\
    Let \( C \) represent a cow in the herd. \\
    Let \( Y \) represent historical milk production data. \\
    
    \[
    \forall C : \textrm{Forecast}(C, Y) \rightarrow \textrm
    {PredictedMilkProduction}(C)
    \]
    
    The system shall provide a forecast of future milk production for each 
    cow \( C \) based on input data \( Y \).
\end{itemize}

\textbf{Specification 3: Herd Retention Likelihood}
\begin{itemize}
    \item \textbf{Description}: The system must predict the likelihood of a cow 
    staying within or leaving the herd, based on its health, productivity, 
    and historical data.
    \item \textbf{Formal Specification}: \\
    Let \( H \) represent a cow's health record. \\
    Let \( P \) represent the predicted probability of retention.
    
    \[
    \forall H : \textrm{RetentionPrediction}(H) \rightarrow P \in [0, 1]
    \]
    
    The system shall compute the retention probability \( P \) for each cow 
    based on its health records and other historical data.
\end{itemize}

\textbf{Specification 4: Data Input Validation}
\begin{itemize}
    \item \textbf{Description}: The system must validate the input data for 
    cows and breeding events to ensure it is accurate and complete before 
    generating predictions.
    \item \textbf{Formal Specification}:
    Let $D$ represent the input data for a cow or breeding event.
    
    $\forall D : \textrm{InputValid}(D) = 
    \left\{
    \begin{array}{ll}
      \textrm{True} & \textrm{if data passes validation checks} \\
      \textrm{False} & \textrm{otherwise}
    \end{array}
    \right.$
    
    The system must ensure that all data $D$ is valid before processing 
    it for predictions.
\end{itemize}

\textbf{Specification 5: Report Generation (TBD)}
\begin{itemize}
    \item \textbf{Description}: The system must generate a report summarizing 
    predictions for breeding success, milk production, and herd retention 
    likelihood.
    \item \textbf{Formal Specification}: \\
    Let \( R \) represent the report generated. \\
    \[
    \forall P, C : \textrm{GenerateReport}(P, C) \rightarrow R
    \]
    The system shall generate a report \( R \) based on the predictions \( P \) 
    and input data \( C \).
\end{itemize}




\section{Look and Feel Requirements}
\subsection{Appearance Requirements}
\textbf{LFR1: Dashboard Display of Predictions}
\begin{itemize}
    \item \textbf{Description}: The system's dashboard shall present the 
    predicted cow traits (e.g., milk production, breeding success) in a 
    structured and organized manner, clearly showing individual predictions 
    for each cow.
    \item \textbf{Rationale}: Farmers need to quickly and easily interpret the 
    predictions without searching through large amounts of data. An organized 
    display ensures that all predictions can be understood at a glance.
    \item \textbf{Fit Criterion}: The system shall display predictions for 
    multiple cows in a table format, with clear labels for each trait, such 
    as milk production and herd retention likelihood.
\end{itemize}
\textbf{LFR2: Text Contrast for Readability}
\begin{itemize}
    \item \textbf{Description}: All text displayed on the system interface 
    shall use a high-contrast color scheme to ensure readability.
    \item \textbf{Rationale}: Farmers and users may access the system in 
    various lighting conditions. High contrast, such as black text on a 
    white background, will ensure clarity.
    \item \textbf{Fit Criterion}: The system shall use a high-contrast 
    color scheme for all text, ensuring that it meets standard readability 
    guidelines under different lighting conditions.
\end{itemize}

\subsection{Style Requirements}
\textbf{LFS1: Consistent Formatting for Input Fields}
\begin{itemize}
    \item \textbf{Description}: All data input fields, such as for entering 
    cow or parental traits, should follow a consistent format with clear labels 
    and input validation.
    \item \textbf{Rationale}: A consistent layout for input fields will 
    minimize errors and ensure ease of use when farmers input or update data.
    \item \textbf{Fit Criterion}: Input fields shall maintain a uniform 
    format, with clear labels and consistent spacing throughout the interface.
\end{itemize}
\textbf{LFS2: Minimalist Design for the Dashboard}
\begin{itemize}
    \item \textbf{Description}: The dashboard interface shall maintain a 
    clean and minimalist design, avoiding unnecessary clutter or decorative 
    elements.
    \item \textbf{Rationale}: A simplified interface will allow farmers to 
    focus on the essential data (predictions) without distractions, ensuring 
    ease of use.
    \item \textbf{Fit Criterion}: Over 80\% of users in a usability test 
    shall report that the dashboard is free from unnecessary elements and 
    easy to navigate.
\end{itemize}


\section{Usability and Humanity Requirements}
\subsection{Ease of Use Requirements}
\textbf{UHR1: Task Completion Efficiency}
\begin{itemize}
    \item \textbf{Description}: The user interface must allow users to complete
    common tasks, such as inputting data and retrieving predictions, within no
    more than 10 minutes.
    \item \textbf{Rationale}: Minimizing the number of steps needed to complete
    tasks will improve user efficiency and reduce frustration, ensuring a
    smoother workflow.
    \item \textbf{Fit Criterion}: During usability testing, all common tasks
    must be completed within 10 minutes for at least 90\% of users.
\end{itemize}
\textbf{UHR2: Quick Onboarding}
\begin{itemize}
    \item \textbf{Description}: New users without prior machine learning
    experience should be able to generate predictions using the system within
    10 minutes, with assistance from help documentation or tooltips.
    \item \textbf{Rationale}: Reducing the learning curve ensures that new
    users can quickly adopt the system, improving overall usability and
    reducing training time.
    \item \textbf{Fit Criterion}: During user trials, at least 80\% of new
    users must be able to generate a prediction within 10 minutes.
\end{itemize}
\subsection{Personalization and Internationalization Requirements}
\textbf{UHR3: Unit Customization}
\begin{itemize}
    \item \textbf{Description}: The system should support switching between
    metric and imperial units for both input and output data related to cow
    metrics.
    \item \textbf{Rationale}: Users may use different measurement systems, so
    supporting unit customization ensures the system is applicable to a broader
    audience and will prevent users from using incorrect inputs due to unit
    mismatches.
    \item \textbf{Fit Criterion}: Usability testing must show that at least
    90\% of users can switch between units and see corresponding changes in
    inputs/outputs.
\end{itemize}
\textbf{UHR4: Language Support}
\begin{itemize}
    \item \textbf{Description}: The system must offer support for English and
    French, ensuring all text elements and documentation are fully translated.
    \item \textbf{Rationale}: Offering English and French options makes the
    system more accessible to all Canadians in Ontario, expanding its usability.
    \item \textbf{Fit Criterion}: Usability testing must confirm that users can
    switch between languages in under 5 seconds, and all UI elements and help
    documentation must be fully translated with no missing text.
\end{itemize}
\subsection{Learning Requirements}
\textbf{UHR5: User Proficiency}
\begin{itemize}
    \item \textbf{Description}: Users should be able to gain proficiency in the
    system's primary functions, such as data input and prediction generation,
    within 2 hours of active use.
    \item \textbf{Rationale}: Ensuring a reasonable learning curve enhances
    adoption rates and minimizes user frustration.
    \item \textbf{Fit Criterion}: At least 80\% of users must pass a proficiency
    test after 2 hours of use, demonstrating the ability to use key system
    functions.
\end{itemize}
\textbf{UHR6: Quick Access to Help}
\begin{itemize}
    \item \textbf{Description}: he system must provide contextual help (e.g.,
    tooltips or tutorials) that can be accessed within 15 seconds for key
    features.
    \item \textbf{Rationale}: Easy access to help reduces confusion and
    minimizes time wasted on seeking external support, improving the overall
    user experience.
    \item \textbf{Fit Criterion}: Usability testing must demonstrate that users
    can activate contextual help for any key feature in under 5 seconds.
\end{itemize}

\subsection{Understandability and Politeness Requirements}
\textbf{UHR7: Clear Output Explanation}
\begin{itemize}
    \item \textbf{Description}: The system must present model predictions in an
    understandable format, including explanations of key factors influencing the
    results.
    \item \textbf{Rationale}: Non-technical users should be able to understand
    how the model arrived at its conclusions, increasing trust and usability of
    the system.
    \item \textbf{Fit Criterion}: At least 80\% of non-technical users must
    indicate that they understand the output and influencing factors after
    using the system.
\end{itemize}
\textbf{UHR8: Informative Error Handling}
\begin{itemize}
    \item \textbf{Description}: Error messages should be clear, non-technical,
    and provide actionable steps for resolution, while being dismissible within
    5 seconds.
    \item \textbf{Rationale}: Polite and informative error handling reduces user
    frustration and helps users quickly recover from mistakes or system issues.
    \item \textbf{Fit Criterion}: Usability testing must show that at least 90\%
    of users can dismiss error messages within 5 seconds and understand how to
    resolve the issue.
\end{itemize}

\subsection{Accessibility Requirements}
\textbf{UHR9: Keyboard Navigation Support}
\begin{itemize}
    \item \textbf{Description}: The system must allow users to navigate and
    interact with all functional elements using only the keyboard.
    \item \textbf{Rationale}: Ensuring that the system is accessible to users
    with motor disabilities or those who prefer keyboard navigation enhances
    inclusivity.
    \item \textbf{Fit Criterion}: Accessibility testing must confirm that all
    interactive elements can be accessed via keyboard shortcuts, with no more
    than 5\% error rate.
\end{itemize}
\textbf{UHR10: WCAG 2.1 AA Compliance}
\begin{itemize}
    \item \textbf{Description}: The system must meet
    \href{https://www.w3.org/TR/WCAG21/}{WCAG 2.1 AA} standards, ensuring that
    elements like color contrast, text size, and screen reader compatibility are
    accessible to users with disabilities.
    \item \textbf{Rationale}: Compliance with established accessibility
    standards ensures that the system is usable by individuals with a wide range
    of abilities, enhancing inclusivity.
    \item \textbf{Fit Criterion}: Automated accessibility testing and user
    trials must confirm that the system meets WCAG 2.1 AA standards, with no
    major violations.
\end{itemize}

\section{Performance Requirements}
\subsection{Speed and Latency Requirements}
\textbf{PFR1: Prediction Response Time}
\begin{itemize}
    \item \textbf{Description}: The system must process input data and return
    predictions within 10 seconds under normal operating conditions.
    \item \textbf{Rationale}: Speed is critical for maintaining smooth user
    experiences and ensuring timely decision-making.
    \item \textbf{Fit Criterion}: 95\% of predictions must be returned within 10
    seconds during testing under normal conditions (100 concurrent users).
\end{itemize}
\textbf{PFR2: User Interface Response Time}
\begin{itemize}
    \item \textbf{Description}: The system's user interface (UI) must respond to
    user interactions, such as button clicks and input field changes, within 0.5
    seconds.
    \item \textbf{Rationale}: Fast UI response times ensure a smooth user
    experience and reduce frustration, especially for data input and navigation
    tasks.
    \item \textbf{Fit Criterion}: During usability testing, the system must
    respond to 90\% of UI interactions within 0.5 seconds.
\end{itemize}

\subsection{Safety-Critical Requirements}
No safety-critical requirements needed.

\subsection{Precision or Accuracy Requirements}
\textbf{PFR3: Prediction Accuracy Threshold}
\begin{itemize}
    \item \textbf{Description}: The machine learning model must achieve a
    minimum prediction accuracy of 85\% when tested against a historical dataset
    of cow offspring features.
    \item \textbf{Rationale}: High prediction accuracy is essential for the
    system to be considered reliable in its intended agricultural context.
    \item \textbf{Fit Criterion}: Testing with a dataset of at least 1,000
    samples must demonstrate a minimum accuracy of 85\%, with results compared
    to known historical outcomes.
\end{itemize}
\textbf{PFR4: Consistency of Prediction Output}
\begin{itemize}
    \item \textbf{Description}: The model must consistently produce the same
    prediction for the same set of input data, with no more than 1\% variation
    in output across multiple trials.
    \item \textbf{Rationale}: Consistency in model outputs ensures reliability,
    preventing users from seeing inconsistent results when using identical
    inputs.
    \item \textbf{Fit Criterion}: Repeated predictions using the same input data
    must produce the same result 99\% of the time in testing across 100 trials.
\end{itemize}

\subsection{Robustness or Fault-Tolerance Requirements}
\textbf{PFR5: System Recovery From Failures}
\begin{itemize}
    \item \textbf{Description}: In the event of a system failure, the system
    must be able to recover within 30 seconds and restore all data from the last
    stable state.
    \item \textbf{Rationale}: Fault tolerance ensures that users do not lose
    data or progress due to system crashes, increasing system reliability.
    \item \textbf{Fit Criterion}: Testing must demonstrate system recovery
    within 30 seconds after simulated failures, with 100\% of data restored to
    its last known state.
\end{itemize}
\textbf{PFR6: Graceful Handling of Invalid Inputs}
\begin{itemize}
    \item \textbf{Description}: The system must handle invalid inputs without
    crashing, providing clear feedback to the user and offering solutions to
    correct the input.
    \item \textbf{Rationale}: Robust handling of invalid inputs ensures system
    stability and improves the user experience by guiding them through input
    corrections.
    \item \textbf{Fit Criterion}: Fault tolerance testing must show that 100\%
    of invalid inputs are flagged, and appropriate error messages are displayed
    without causing system crashes.
\end{itemize}
\textbf{PFR7: Load Balancing for Heavy Traffic}
\begin{itemize}
    \item \textbf{Description}: The system must automatically distribute traffic
    across servers in case of high load, ensuring stable performance for all
    users.
    \item \textbf{Rationale}: Robust load balancing prevents system downtime or
    performance degradation during periods of heavy usage.
    \item \textbf{Fit Criterion}: Load tests must demonstrate that the system
    can handle spikes in traffic (up to 150\% of normal load) without response
    times exceeding 10 seconds.
\end{itemize}

\subsection{Capacity Requirements}
\textbf{PFR8: User Load Capacity}
\begin{itemize}
    \item \textbf{Description}: The system must be able to handle at least 500
    concurrent users without performance degradation.
    \item \textbf{Rationale}: Ensuring high capacity allows for scalability and
    supports widespread adoption of the system by large user bases.
    \item \textbf{Fit Criterion}: Load testing must confirm that the system can
    support at least 500 concurrent users while maintaining prediction response
    times of under 15 seconds.
\end{itemize}

\subsection{Scalability or Extensibility Requirements}
\textbf{PFR9: Modular Design}
\begin{itemize}
    \item \textbf{Description}: The system’s architecture must be modular,
    allowing new features such as additional machine learning models, data
    types, or metrics to be integrated without major redesign.
    \item \textbf{Rationale}: A modular design enables easy expansion as new
    business requirements emerge, increasing the system’s longevity and
    adaptability.
    \item \textbf{Fit Criterion}: Documentation and testing must confirm that
    new features can be added or updated without affecting the core
    functionality of existing modules.
\end{itemize}
\textbf{PFR10: Vertical Scaling for Increased Demand}
\begin{itemize}
    \item \textbf{Description}: The system must be able to scale vertically to
    handle increased demand by adding more computational resources without
    redesigning the architecture.
    \item \textbf{Rationale}: Vertical scalability ensures that as demand grows,
    the system can continue to function efficiently by utilizing additional
    hardware resources.
    \item \textbf{Fit Criterion}: Stress tests must demonstrate that adding
    computational resources (e.g., CPUs, memory) results in proportional
    performance gains under increased loads.
\end{itemize}

\subsection{Longevity Requirements}
\textbf{PFR11: Code Maintainability}
\begin{itemize}
    \item \textbf{Description}: The system’s codebase must be well-documented,
    following the set coding standards to ensure that it can be maintained and
    extended over the next 5 years without requiring significant rework.
    \item \textbf{Rationale}: Ensuring code maintainability reduces future
    technical debt and allows for easier updates and feature expansions.
    \item \textbf{Fit Criterion}: Code reviews and external audits must confirm
    that the code adheres to industry-standard practices and includes sufficient
    documentation for future maintainers.
\end{itemize}
\textbf{PFR12: Support for Future Technologies}
\begin{itemize}
    \item \textbf{Description}: The system must be designed to support the
    integration of future technologies, including newer machine learning models
    or external data sources, without requiring a full system overhaul.
    \item \textbf{Rationale}: Designing for longevity ensures that the system
    can evolve with advances in technology, preserving its relevance and
    utility.
    \item \textbf{Fit Criterion}:  System architecture and documentation must
    demonstrate that the system can accommodate integration of future
    technologies with minimal redesign.
\end{itemize}


\section{Operational and Environmental Requirements}
\subsection{Expected Physical Environment}
\textbf{OER1: Network Stability}
\begin{itemize}
    \item \textbf{Description}: The system must operate reliably over both
    stable and fluctuating internet connections.
    \item \textbf{Rationale}: Given that the CATTLEytics system may be deployed
    in rural farm settings with less stable internet, the system needs to handle
    both stable and intermittent connectivity.
    \item \textbf{Fit Criterion}: The system must function with no critical
    failures in environments with intermittent connectivity and maintain
    prediction response times within 25\% of normal operation during network
    fluctuations.
\end{itemize}

\subsection{Wider Environment Requirements}
\textbf{OER2: Sustainability in Hosting}
\begin{itemize}
    \item \textbf{Description}: The system’s cloud-based infrastructure must
    prioritize sustainability, with at least 50\% of its hosting powered by
    renewable energy sources.
    \item \textbf{Rationale}: Environmental sustainability is a growing
    priority, and using renewable energy sources can reduce the system’s carbon
    footprint.
    \item \textbf{Fit Criterion}: Hosting provider documentation must show that
    at least 50\% of the energy used in hosting services comes from renewable
    sources.
\end{itemize}

\subsection{Requirements for Interfacing with Adjacent Systems}
\textbf{OER3: API Compatability with CATTLEytics}
\begin{itemize}
    \item \textbf{Description}: The system must provide a well-documented API to
    interface with CATTLEytics' existing application and third-party tools,
    supporting data exchange in standard formats.
    \item \textbf{Rationale}: Ensuring smooth data exchange between the
    prediction model and adjacent systems, such as farm management platforms, is
    essential for integration and workflow efficiency.
    \item \textbf{Fit Criterion}: API testing must confirm successful data
    exchange with the CATTLEytics platform and third-party tools, with 100\% of
    test cases passing.
\end{itemize}
\textbf{OER4: Seamless Data Transfer}
\begin{itemize}
    \item \textbf{Description}: The system must ensure that data transfers
    between the prediction model and adjacent systems are fault-tolerant, with
    automatic retries for failed transmissions.
    \item \textbf{Rationale}: Seamless data integration is essential to prevent
    loss of data during transmission, ensuring that all related systems remain
    synchronized and accurate.
    \item \textbf{Fit Criterion}: In testing, the system must recover from 100\%
    of simulated transmission failures by automatically retrying data transfers
    without user intervention.
\end{itemize}

\subsection{Productization Requirements}
\textbf{OER5: Automated Updates}
\begin{itemize}
    \item \textbf{Description}: The system must support automated updates,
    ensuring that all future software patches and improvements can be deployed
    without user intervention.
    \item \textbf{Rationale}: Automated updates ensure that the system remains
    up-to-date with the latest features and security patches without requiring
    manual intervention, reducing downtime and maintenance costs.
    \item \textbf{Fit Criterion}: Testing must confirm that updates are
    automatically deployed in 100\% of test cases, with no need for user
    interaction during the update process.
\end{itemize}

\subsection{Release Requirements}
\textbf{OER6: Testing and Revision Cycles}
\begin{itemize}
    \item \textbf{Description}: The system must undergo a formal testing phase,
    involving key stakeholders (e.g., farm operators, internal CATTLEytics team
    members), with feedback collected to identify potential issues before final
    release.
    \item \textbf{Rationale}: Testing with actual users ensures that real-world
    use cases and unforeseen issues are identified and resolved, improving
    overall product quality.
    \item \textbf{Fit Criterion}: Feedback from testers must be collected and
    analyzed, with all critical issues resolved before the final product
    release. At least 10 users must be involved in this testing phase.
\end{itemize}
\textbf{OER7: Documentation for End Users and Developers}
\begin{itemize}
    \item \textbf{Description}: The system must include comprehensive
    documentation for both end users (features, usage) and developers (API,
    integration, system architecture) prior to release.
    \item \textbf{Rationale}: Clear and thorough documentation ensures that both
    users and developers can effectively utilize and maintain the system,
    reducing support costs and improving product adoption.
    \item \textbf{Fit Criterion}: Documentation must pass an internal review,
    with at least 90\% of users and developers rating the documentation as
    “sufficient” or better in post-release surveys.
\end{itemize}
\textbf{OER8: Release Versioning and Rollback Support}
\begin{itemize}
    \item \textbf{Description}: Each system release must be versioned, with the
    ability to rollback to a previous stable version if critical issues arise
    post-release.
    \item \textbf{Rationale}: Versioning and rollback capabilities are essential
    for ensuring stability and addressing issues that might emerge in new
    releases, minimizing system downtime and user impact.
    \item \textbf{Fit Criterion}: All releases must be properly versioned, and
    rollback testing must confirm that previous versions can be restored without
    data loss or functionality degradation.
\end{itemize}

\section{Maintainability and Support Requirements}
\subsection{Maintenance Requirements}
\textbf{MSR1: Scheduled Maintenance Windows}
\begin{itemize}
    \item \textbf{Description}: The system must support scheduled maintenance
    windows, during which the application can be updated or maintained without
    impacting user activity.
    \item \textbf{Rationale}: Planned maintenance reduces the risk of unexpected
    downtime and allows system updates to be performed in a controlled manner.
    \item \textbf{Fit Criterion}: Testing must demonstrate that the system can
    be taken offline for maintenance and restored within a scheduled
    maintenance window of no more than 4 hours, with no data loss.
\end{itemize}
\textbf{MSR2: Automated Testing for Releases}
\begin{itemize}
    \item \textbf{Description}: The system must incorporate automated testing to
    ensure that all new maintenance releases pass unit, integration, and
    regression tests before deployment.
    \item \textbf{Rationale}: Automated testing ensures that updates or bug
    fixes do not introduce new errors, reducing the time and effort required for
    manual testing.
    \item \textbf{Fit Criterion}: Automated tests must cover at least 80\% of
    the codebase, with no critical failures in tests run before each maintenance
    release.
\end{itemize}

\subsection{Supportability Requirements}
\textbf{MSR3: Multi-tier Support Structure}
\begin{itemize}
    \item \textbf{Description}: The system must support a multi-tier support
    structure, offering different levels of assistance, from basic user help
    (FAQs) to advanced technical support (dedicated IT assistance).
    \item \textbf{Rationale}: A tiered support system ensures that users receive
    the appropriate level of assistance based on the complexity of the issue,
    improving resolution efficiency.
    \item \textbf{Fit Criterion}: Support logs must show that 90\% of issues are
    resolved within the expected timeframe for each support tier, with at least
    75\% of basic issues resolved through self-service (e.g., FAQs).
\end{itemize}
\textbf{MSR4: Comprehensive Error Logging}
\begin{itemize}
    \item \textbf{Description}: The system must implement comprehensive error
    logging, capturing detailed error messages for all exceptions and
    performance issues.
    \item \textbf{Rationale}: Detailed error logs help support teams quickly
    diagnose and resolve problems, ensuring that users experience minimal
    downtime.
    \item \textbf{Fit Criterion}: Error logs must capture 100\% of critical
    errors during testing, with logs providing detailed information sufficient
    for issue resolution in 90\% of cases.
\end{itemize}

\subsection{Adaptability Requirements}
\textbf{MSR5: Plug-in Architecture for Future Features}
\begin{itemize}
    \item \textbf{Description}: The system must be built using a plug-in
    architecture, allowing new features or modules (e.g., additional prediction
    algorithms) to be added without impacting core functionality.
    \item \textbf{Rationale}: A plug-in architecture allows the system to be
    extended easily in the future, enabling it to adapt to evolving business or
    technical requirements.
    \item \textbf{Fit Criterion}: At least one new feature or module must be
    added as a plug-in during testing, with no impact on the core system or
    existing features.
\end{itemize}

\section{Security Requirements}
\subsection{Access Requirements}
\textbf{SCR1: Role-Based Access Control (RBAC)}
\begin{itemize}
    \item \textbf{Description}: The system must implement role-based access
    control to restrict access based on user roles (e.g., administrator, data
    analyst, external user).
    \item \textbf{Rationale}: RBAC ensures that only authorized users can access
    sensitive parts of the system, reducing the risk of unauthorized access or
    manipulation.
    \item \textbf{Fit Criterion}: Access tests must verify that users with
    different roles can only access the functionalities or data assigned to
    their role, with 100\% compliance in role-based access rules during testing.
\end{itemize}
\textbf{SCR2: Session Timeout}
\begin{itemize}
    \item \textbf{Description}: The system must automatically log out users
    after 15 minutes of inactivity to prevent unauthorized access via unattended
    sessions.
    \item \textbf{Rationale}: Automatic session timeout reduces the likelihood
    of unauthorized access due to forgotten or unattended active sessions.
    \item \textbf{Fit Criterion}: Testing must show that all user sessions are
    automatically terminated after 15 minutes of inactivity, with users being
    required to log back in to regain access.
\end{itemize}

\subsection{Integrity Requirements}
\textbf{SCR3: Data Validation on Input}
\begin{itemize}
    \item \textbf{Description}: The system must implement robust input
    validation to prevent invalid or malicious data from being entered and
    processed by the application.
    \item \textbf{Rationale}: Proper data validation protects against injection
    attacks (e.g., SQL injection) and ensures that only valid data is processed.
    \item \textbf{Fit Criterion}: Penetration testing must confirm that no
    invalid or malicious data bypasses input validation mechanisms in 100\% of
    test cases.
\end{itemize}
\textbf{SCR4: Transaction Atomicity}
\begin{itemize}
    \item \textbf{Description}: The system must ensure that all data
    transactions are atomic, meaning they are either fully completed or fully
    rolled back in case of an error.
    \item \textbf{Rationale}: Atomic transactions ensure data integrity by
    preventing partial updates that could corrupt data or leave the system in an
    inconsistent state.
    \item \textbf{Fit Criterion}: During testing, all simulated transaction
    failures must result in a full rollback with no partial updates to the
    database.
\end{itemize}
\textbf{SCR5: Data Consistency Across Systems}
\begin{itemize}
    \item \textbf{Description}: The system must maintain data consistency across
    all integrated systems, ensuring that updates or changes are reflected
    accurately across the board.
    \item \textbf{Rationale}: Consistent data across integrated systems prevents
    discrepancies that could lead to erroneous predictions or reports.
    \item \textbf{Fit Criterion}: Consistency checks during testing must verify
    that updates in one system are correctly reflected across all other systems
    in 100\% of test cases.
\end{itemize}

\subsection{Privacy Requirements}
\textbf{SCR6: Data Encryption}
\begin{itemize}
    \item \textbf{Description}: All sensitive data, including personal
    information and prediction results, must be encrypted both at rest and in
    transit using industry-standard encryption protocols.
    \item \textbf{Rationale}: Encrypting sensitive data ensures that even if the
    data is intercepted or accessed without authorization, it remains unreadable
    and secure.
    \item \textbf{Fit Criterion}: Testing must confirm that all sensitive data
    is encrypted with no plaintext exposure during storage or transmission,
    potentially using protocols such as AES-256 for encryption at rest and TLS
    1.2+ for encryption in transit.
\end{itemize}

\subsection{Audit Requirements}
\textbf{SCR7: Data Access Monitoring}
\begin{itemize}
    \item \textbf{Description}: The system must monitor and log all access to
    sensitive data, with alerts triggered for any unauthorized access attempts.
    \item \textbf{Rationale}: Continuous monitoring of data access ensures that
    any unauthorized or suspicious activity is detected early, preventing data
    breaches.
    \item \textbf{Fit Criterion}: Testing must confirm that unauthorized access
    attempts trigger an alert in real time, and all data access is logged and
    available for review in audit logs.
\end{itemize}
\textbf{SCR8: Audit Logging of Sensitive Operations}
\begin{itemize}
    \item \textbf{Description}: The system must log all sensitive operations
    (e.g., data access, user account changes, system configuration changes) for
    auditing and compliance purposes.
    \item \textbf{Rationale}: Audit logs allow tracking of who did what and
    when, providing transparency and accountability for system changes and data
    access.
    \item \textbf{Fit Criterion}: Testing must verify that all sensitive
    operations are logged, with each log entry containing a timestamp, user ID,
    and details of the action performed.
\end{itemize}
\textbf{SCR9: Audit Log Retention Policy}
\begin{itemize}
    \item \textbf{Description}: The system must retain audit logs for a minimum
    of 1 year, ensuring that historical data can be reviewed for compliance and
    forensic investigations.
    \item \textbf{Rationale}: Retaining audit logs for an adequate period is
    essential for compliance with regulatory requirements and for investigating
    any incidents that occur.
    \item \textbf{Fit Criterion}: Testing must confirm that audit logs are
    stored securely and accessible for at least 1 year, with no loss of log data
    during the retention period.
\end{itemize}

\subsection{Immunity Requirements}
\textbf{SCR10: Resilience Against Denial-of-Service (DoS) Attacks}
\begin{itemize}
    \item \textbf{Description}: The system must be resilient to 
    denial-of-service (DoS) attacks, with automatic measures in place to 
    mitigate the impact of such attacks.
    \item \textbf{Rationale}: DoS attacks can disrupt system availability and
    cause significant downtime, making resilience essential for maintaining
    uninterrupted service.
    \item \textbf{Fit Criterion}: Simulated DoS attacks must show that the
    system remains available for 95\% of legitimate users, with automated
    scaling or mitigation strategies successfully limiting the attack's impact.
\end{itemize}
\textbf{SCR11: Recovery From Cyberattacks}
\begin{itemize}
    \item \textbf{Description}: The system must have a defined recovery protocol
    that ensures operations can be restored within 4 hours in the event of a
    successful cyberattack.
    \item \textbf{Rationale}: In case of a security breach, a fast recovery
    process minimizes downtime and limits the damage caused by the attack.
    \item \textbf{Fit Criterion}: Disaster recovery tests must confirm that the
    system can be fully restored within 4 hours after a simulated breach, with
    no loss of critical data.
\end{itemize}


\section{Cultural Requirements}
\subsection{Cultural Requirements}
\textbf{CR1: Language and Units}
\begin{itemize}
    \item \textbf{Description}: The system must primarily use English for all
    user interfaces and documentation. All data and measurements should follow
    Canadian standards, including liters for milk production, kilograms for
    weight, hectares for land area, Celsius for temperature, and metric tons for
    large quantities.
    \item \textbf{Rationale}: Using standardized language and units ensures the
    product is accessible and relevant to Canadian dairy farmers and adheres to
    local agricultural practices.
    \item \textbf{Fit Criterion}: User interface testing must confirm that all
    text appears in English, and data entries for milk, weight, temperature, and
    land area are consistently displayed in the appropriate Canadian units.
\end{itemize}

\section{Compliance Requirements}
\subsection{Legal Requirements}
\textbf{LR1: Code of Practice for Dairy Cattle}
\begin{itemize}
    \item \textbf{Description}: The project must comply with the
    \href{https://www.nfacc.ca/codes-of-practice/dairy-cattle}{Code of Practice
    for the Care and Handling of Dairy Cattle}, a regulation by the Canadian
    government for ethical treatment and welfare of dairy cattle.
    \item \textbf{Rationale}: This code outlines the mandatory guidelines for
    ensuring the health and welfare of dairy cattle in Canada. Any
    recommendations or outputs from the machine learning model must align with
    these regulations.
    \item \textbf{Fit Criterion}: Testing must confirm that all suggested
    management actions and predictions follow the guidelines outlined in the
    Code of Practice, ensuring ethical treatment of the cattle.
\end{itemize}

\textbf{LR2: PIPEDA Compliance}
\begin{itemize}
    \item \textbf{Description}: The project must comply with
    \href{https://laws-lois.justice.gc.ca/pdf/p-8.6.pdf}{PIPEDA} (Personal
    Information Protection and Electronic Documents Act) when handling dairy
    farmer information. This includes personal details like contact and
    financial information.
    \item \textbf{Rationale}: PIPEDA governs how personal information is
    collected, stored, and shared, ensuring the privacy of individuals
    associated with the project.
    \item \textbf{Fit Criterion}: Testing must confirm that any dairy farmer
    data is stored securely, access is restricted, and no personal data is
    shared without consent, in full compliance with PIPEDA regulations.
\end{itemize}
\subsection{Standards Compliance Requirements}
\begin{itemize}
	\item There are no specific standards for collecting dairy farming data in
	      this project. All relevant aspects of data collection and handling are
	      already covered under Legal Requirements, specifically in compliance
	      with PIPEDA for managing sensitive information about dairy farmers, 
          and the Code of Practice for the Care and Handling of Dairy Cattle for
	      ensuring the welfare of the animals.
	\item For coding standards, the project will adhere to PEP8 to ensure
	      consistent and readable Python code. More information on PEP8 can be
	      found \href{https://peps.python.org/pep-0008/}{here}.
\end{itemize}

\section{Open Issues}
\begin{itemize}
	\item \textbf{Data Availability and Quality:} The accuracy of predictions will
	      heavily depend on the quality and completeness of the data obtained from
	      CATTLEytics and Lactanet. Inconsistent or missing data might affect the
	      performance of the model.
	      
	\item \textbf{Model Accuracy:} The machine learning model may need to be
	      fine-tuned multiple times to achieve high accuracy in predicting cow
	      traits. This requires testing with diverse datasets to ensure the model
	      generalizes well.
	      
	\item \textbf{User Interface Usability:} The graphical representation of the
	      family tree and predicted traits needs to be intuitive and user-friendly
	      for farmers with varying levels of technical skill. Determining the best
	      design and ensuring it meets users' needs could take time.
	      
	\item \textbf{Integration with CATTLEytics:} Seamlessly integrating the tool
	      into the existing CATTLEytics system without causing disruptions or
	      requiring major system changes could be technically challenging.
	      
	\item \textbf{Regulatory Compliance:} Ensuring that the predictions and
	      recommendations made by the model comply with Canadian regulations for
	      dairy farming (Code of Practice for the Care and Handling of Dairy
	      Cattle) will require thorough review and potential adjustments during
	      development.
	      
	\item \textbf{Model Interpretability:} Farmers may need clear explanations for
	      how predictions are made to trust and use the tool effectively. Ensuring
	      the model’s predictions are explainable is an open issue.
	      
	\item \textbf{Performance Considerations:} The tool needs to be efficient and
	      scalable, handling large amounts of data without significant lag or
	      performance issues, especially as it gets adopted by multiple farms.
\end{itemize}

\section{Off-the-Shelf Solutions}
\subsection{Ready-Made Products}
\begin{itemize}
	\item There are no fully ready-made products that address the predictive
	      capabilities being developed in this project. While tools like Lactanet
	      provide dairy farm data, they do not offer predictive models based on
	      genetic and health data. Lactanet data will be used primarily for
	      training the custom machine learning model.
\end{itemize}
\subsection{Reusable Components}
\begin{itemize}
	\item Machine learning libraries, such as PyTorch or TensorFlow, will be
	      utilized to develop the custom AI model for cow trait prediction.
	      Additionally, front-end libraries such as D3.js or React Tree
	      Visualization libraries could be considered for visualizing the
	      family-tree diagrams.
\end{itemize}
\subsection{Products That Can Be Copied}
\begin{itemize}
	\item There are no existing products to be copied for this project. However,
	      open-source family-tree visualization tools might serve as inspiration
	      for the graphical aspects of the project.
\end{itemize}

\section{New Problems}
\subsection{Effects on the Current Environment}
\begin{itemize}
	\item Introducing this system could change how farmers currently select or 
    evaluate herd performance. Some may resist adopting new technology due to 
    unfamiliarity.
\end{itemize}
\subsection{Effects on the Installed Systems}
\begin{itemize}
	\item The project will be integrated into the existing Cattleytics software,
	      which is already used to manage dairy farms. The machine learning tool
	      will act as an additional module within Cattleytics, allowing farmers 
          to visualize the family tree of cows and predict future traits based 
          on genetic data. Seamless integration with the current system will be
	      prioritized to ensure smooth adoption and ease of use.
\end{itemize}
\subsection{Potential User Problems}
\begin{itemize}
	\item Users may face difficulties interpreting complex AI model outputs, so
	      ensuring the tool’s recommendations are easy to understand is key.
\end{itemize}
\subsection{Limitations in the Anticipated Implementation Environment That May
Inhibit the New Product}
\begin{itemize}
	\item The tool will need to function effectively on standard farm computing
	      systems, which may have limited processing power or internet
	      connectivity.
\end{itemize}
\subsection{Follow-Up Problems}
\begin{itemize}
	\item Continuous updates may be needed to improve the model based on 
    feedback from farmers. Future updates may also need to address changes in  
    farming practices
\end{itemize}

\section{Tasks}
\subsection{Project Planning}

The project will be scheduled to follow the deliverables deadlines as outlined
in the SFWRENG 4G06 course outline.

\begin{table}[h]
	\centering
	\caption{Project Documentation Tasks}
	\vspace{5pt}
	\begin{tabular}{|p{0.15\textwidth}|p{0.47\textwidth}|p{0.3\textwidth}|}
		\hline
		\textbf{Phase} & \textbf{Task}                      & \textbf{Due Date}
		\\
		\hline
		Phase 1        & Hazard Analysis                    & October 23, 2024
		\\
		\cline{2-3}    & Verification and Validation Plan   & November 1, 2024
		\\
		\cline{2-3}    & Proof of Concept Demonstration     & November 11--22, 
        2024
		\\
		\cline{2-3}    & Design Documentat                  & January 15, 2025
		\\
		\cline{2-3}    & Revision 0 Demonstration           & February 3--14, 
        2025
		\\
		\hline
		Phase 2        & Verification and Validation Report & March 7, 2025
		\\
		\cline{2-3}    & Final Demonstration                & March 24--30, 2025
		\\
		\cline{2-3}    & Final Documentation                & April 2, 2025
		\\
		\hline
	\end{tabular}

	\label{project_tasks}
\end{table}

\subsection{Planning of the Development Phases}
The development of the project will be conducted in four primary phases:

\textbf{1. Data Acquisition and Initial Integration:} This initial phase focuses
on acquiring, cleaning, and integrating historical dairy farm data into the
system. Activities include:
\begin{itemize}
    \item Collecting and preprocessing datasets from CATTLEytics and Lactanet to
    ensure data quality and consistency.
    \item Developing scripts to handle data normalization, unit conversion
    (e.g., to liters and kilograms), and encryption for compliance with PIPEDA.
    \item Implementing an initial data pipeline to allow seamless integration
    between the existing CATTLEytics system and the new module.
\end{itemize}
This phase will conclude with a Proof of Concept (POC) demonstration, where the
pipeline and initial data handling capabilities will be validated.

\textbf{2. Development of Predictive Model and Core Functionality:} The second
phase is dedicated to developing and integrating the core machine learning model
responsible for predicting key outcomes such as breeding success rates. This
phase involves:
\begin{itemize}
    \item Experimenting with various algorithms (e.g., logistic regression,
    decision trees, neural networks) using Python libraries such as
    Scikit-learn, TensorFlow, or PyTorch to determine the most suitable approach
    for our data.
    \item Developing the model's training, validation, and testing processes,
    ensuring the system is optimized for accuracy and performance.
    \item Implementing backend APIs for data retrieval and model output
    generation to support the frontend development in the next phase.
\end{itemize}
This phase will culminate with an evaluation of the model's initial accuracy and
robustness, and any necessary refinements will be documented for the next
sprint.

\textbf{3. System Integration and Frontend Development:} In the third phase, the
focus shifts to building the user-facing components and integrating the machine
learning model into the existing CATTLEytics system. Activities include:
\begin{itemize}
    \item Developing a React-based frontend interface that allows farmers to
    input data, visualize predictions, and interpret results in an intuitive
    manner.
    \item Implementing interactive visualizations using libraries like D3.js to
    display herd health trends, genetic family trees, and predictive insights.
    \item Integrating the frontend with the backend APIs, ensuring seamless data
    flow and responsiveness across the system.
    \item Initial system testing to verify that the user interface works well
    with the machine learning outputs and data pipeline.
\end{itemize}
This phase will end with a functional demo of the frontend and backend
integration, allowing for early user feedback and adjustments.

\textbf{4. Testing, Deployment, and Refinement:} The final phase focuses on
refining the system, conducting extensive testing, and preparing for deployment.
Key activities include:
\begin{itemize}
    \item Performing unit, integration, and system-level testing to validate the
    accuracy, performance, and reliability of the predictive model and frontend
    interface.
    \item Setting up a Continuous Integration/Continuous Deployment (CI/CD)
    pipeline using GitHub Actions to automate testing and deployment to a
    staging environment.
    \item Conducting user testing sessions with industry stakeholders to gather
    feedback, refine features, and enhance the usability of the system.
    \item Finalizing user documentation, training materials, and preparing the
    system for deployment to the production environment.
\end{itemize}
This phase will conclude with a complete system demonstration and the final
handoff to stakeholders, ensuring the solution meets all requirements and is
ready for the EXPO presentation.

All phases will run concurrently with documentation development to meet capstone
deliverables and ensure that non-functional requirements, such as performance
and security, are integrated throughout the development process. Each phase will
have some overlap to allow flexibility and responsiveness to evolving
requirements.


\section{Migration to the New Product}
\subsection{Requirements for Migration to the New Product}
The new predictive module that is being developed for this project will need to
be integrated into the existing CATTLEytics software. Key requirements include:
\begin{itemize}
    \item Ensuring compatibility of the new module with existing CATTLEytics
    data formats and APIs.
    \item Ensuring that the frontend interface seamlessly integrates with the
    existing user interface of CATTLEytics.
    \item Developing a rollback plan to ensure that, if issues occur during
    migration, the system can revert to the previous stable state.
    \item Testing the migration process on a staging environment to validate
    functionality before deployment.
\end{itemize}

\subsection{Data That Has to be Modified or Translated for the New System}
This section will be completed once the relevant data model details are
available.

\section{Costs}
The goal for this project is to minimize costs by leveraging open-source tools
and resources wherever possible. The primary development tools, including
Python, TensorFlow, PyTorch, React, and GitHub for version control, are
open-source and free to use. Continuous Integration will be managed through
GitHub Actions, which offers a free tier suitable for our needs. 

Any additional datasets needed from third-party providers (e.g., Lactanet) may
incur small fees, but these will be minimized through the use of open or freely
available datasets when possible. The team at CATTLEytics will provide access to
a dataset that they have scraped from Lactanet.

At this time, no other significant direct monetary costs are anticipated. The
project plan remains flexible to adapt if unforeseen expenses arise during
development, but the team will prioritize cost-effective and open-source
solutions whenever feasible.

\section{User Documentation and Training}
\subsection{User Documentation Requirements}
The user documentation for the system will be concise and user-friendly, aimed
at ensuring farmers and stakeholders can easily understand and utilize the tool.
The documentation will include:
\begin{itemize}
  \item Step-by-step instructions on accessing and utilizing the platform.
  \item Instructions on interpreting reports generated by the system.
  \item A troubleshooting section to address common issues.
\end{itemize}

\subsection{Training Requirements}
No formal training will be required as the tool is designed to be intuitive and
straightforward. The provided documentation will be sufficient for users to
operate the system without additional assistance.

\section{Waiting Room}
This section will be updated to highlight requirements that need to be put on
hold due to time constraints or other factors.

\section{Ideas for Solution}

The solution calls for a ML model that can predict the bredding success rate of 
desired traits for offsprings based on the sire and dam.

\begin{itemize}
    \item \textbf{Model Type:} A supervised learning model will be developed to 
    predict the traits of offspring based on the input features from the sire 
    and dam. This model will analyze the relationships between parental traits 
    and the resultant traits of the offspring.

    \item \textbf{Learning Style:} The learning style will be supervised 
    learning, which involves training the model on labeled data (where the 
    traits of offspring are known) to enable it to learn the underlying 
    patterns and relationships.

    \item \textbf{Algorithm Selection:} Suitable algorithms for this task may 
    include:
    \begin{itemize}
        \item \textbf{Random Forest:} This ensemble method is robust against 
        overfitting and can handle both linear and non-linear relationships 
        effectively, making it suitable for the prediction of complex traits.
        \item \textbf{Gradient Boosting Machines (GBM):} This method builds 
        predictive models iteratively and is known for its high accuracy, 
        making it a strong candidate for capturing intricate relationships in 
        the dataset.
    \end{itemize}

    \item \textbf{Output:} The model will provide predictions regarding the 
    likelihood of specific traits in the offspring based on the input traits of 
    the sire and dam, enabling farmers to make informed breeding decisions.
\\
As the project progresses, and we have access to the dataset and the variables 
it holds, we can propose solutions that will match the final solution.
\end{itemize}




\newpage{}
\section*{Appendix --- Reflection}

1. What knowledge and skills will the team collectively need to acquire to
successfully complete this capstone project?  Examples of possible knowledge
to acquire include domain specific knowledge from the domain of your
application, or software engineering knowledge, mechatronics knowledge or
computer science knowledge.  Skills may be related to technology, or writing,
or presentation, or team management, etc.  You should look to identify at
least one item for each team member. \\

\begin{itemize}
    \item \textbf{Domain Knowledge on Cow Traits}:
    \begin{itemize}
        \item Understanding the key genetic and health traits in dairy cows, 
        such as lameness, milk production, fertility, and longevity.
        \item Knowledge of how sire and dam traits influence offspring 
        characteristics.
    \end{itemize}

    \item \textbf{Machine Learning (ML)}:
    \begin{itemize}
        \item Supervised learning techniques, including algorithms like Random 
        Forest and Gradient Boosting Machines (GBM).
        \item Data preprocessing techniques, including handling missing data, 
        feature selection, and feature engineering.
        \item Model evaluation methods (accuracy, precision, recall, etc.) and 
        cross-validation.
    \end{itemize}

    \item \textbf{Data Management and Processing}:
    \begin{itemize}
        \item Skills in handling large datasets, including data cleaning, 
        manipulation, and integration from multiple sources.
        \item Knowledge of database systems and how to manage, query, and store 
        data efficiently for ML tasks.
    \end{itemize}

    \item \textbf{Software Development and Integration}:
    \begin{itemize}
        \item Proficiency in programming languages such as Python, including 
        libraries like TensorFlow and Pandas.
        \item Experience with integrating machine learning models into an 
        existing platforms.
    \end{itemize}

    \item \textbf{Project Management and Communication}:
    \begin{itemize}
        \item Team management skills, including task delegation, timeline 
        management, and collaboration.
        \item Proficiency in writing technical documentation, delivering 
        presentations, and clearly communicating technical concepts to 
        stakeholders.
    \end{itemize}
\end{itemize}



2. For each of the knowledge areas and skills identified in the previous
question, what are at least two approaches to acquiring the knowledge or
mastering the skill?  Of the identified approaches, which will each team
member pursue, and why did they make this choice? \\


\begin{itemize}
    \item \textbf{Approach 1}: \textit{Consult with Experts in the Dairy 
    Industry} \\
    The team will host meetings with those within the dairy industry. This 
    would include our industy stakeholder, CATTLEytics. They can provide 
    valuable insight that would not be obtained with online searched. 
    \item \textbf{Approach 2}: \textit{Research Online Resources} \\
    The team could access academic papers and industry reports related to dairy 
    genetics and animal health. Websites like Lactanet or academic journals 
    could be excellent sources for this research.
\end{itemize}

\textbf{Team Member Assignment}:
\begin{itemize}
    \item \textbf{Krish} will pursue \textit{consulting with industry experts}. 
    He is the Liason of the team and he has established a strong connection 
    with the CATTLEytics team.
    \item \textbf{Krish} will also focus on \textit{researching online 
    resources}. He has resarch experience and is skilled in researching 
    academic literature and presenting that information in a digestable manner.
\end{itemize}

\textbf{Machine Learning (ML) Models and Training Techniques}

\begin{itemize}
    \item \textbf{Approach 1}: \textit{Online Courses and Tutorials} \\
    Platforms like \textit{Coursera}, \textit{edX}, or \textit{Udemy} offer ML 
    courses that cover everything from basic concepts to more advanced 
    techniques, such as supervised learning and model evaluation.
    \item \textbf{Approach 2}: \textit{Hands-On Practice with ML Tools} \\
    The team can gain experience by working with real ML tools like \textit
    {TensorFlow}. Practicing with sample datasets, building models, and running 
    them through the pipeline will give hands-on knowledge of how to build and 
    fine-tune models.
\end{itemize}

\textbf{Team Member Assignment}:
\begin{itemize}
    \item \textbf{Martin} will focus on \textit{online courses} as he as an 
    interest in learning more about ML.
    \item \textbf{Harsh} will focus on \textit{hands-on practice} since she has 
    experience with ML models.
\end{itemize}

\textbf{Data Management and Processing}

\begin{itemize}
    \item \textbf{Approach 1}: \textit{Tutorials on Data Science Tools} \\
    The team can participate in workshops or online tutorials that cover tools 
    like \textit{Pandas}, \textit{SQL}, and \textit{NumPy} for data handling, 
    cleaning, and manipulation.
    \item \textbf{Approach 2}: \textit{Guidence from Mentors} \\
    Engaging with a data engineer mentor who has experience in managing large 
    datasets could provide practical guidance. 
\end{itemize}

\textbf{Team Member Assignment}:
\begin{itemize}
    \item \textbf{Aryan} will focus on \textit{data sciecne tutorials}, as he 
    has interest in mastering data tools.
    \item \textbf{Shazim} will pursue \textit{learning from a mentor}. Ideally, 
    this mentor would be from CATTLEytics.
\end{itemize}

\end{document}