% THIS DOCUMENT IS FOLLOWS THE VOLERE TEMPLATE BY Suzanne Robertson and James Robertson
% ONLY THE SECTION HEADINGS ARE PROVIDED
%
% Initial draft from https://github.com/Dieblich/volere
%
% Risks are removed because they are covered by the Hazard Analysis
\documentclass[12pt]{article}

\usepackage{booktabs}
\usepackage{tabularx}
\usepackage{hyperref}
\hypersetup{
    bookmarks=true,         % show bookmarks bar?
      colorlinks=true,      % false: boxed links; true: colored links
    linkcolor=red,          % color of internal links (change box color with linkbordercolor)
    citecolor=green,        % color of links to bibliography
    filecolor=magenta,      % color of file links
    urlcolor=cyan           % color of external links
}

\newcommand{\lips}{\textit{Insert your content here.}}
\newcommand{\mname}[1]{\mbox{\sf #1}}

%% Comments

\usepackage{color}

\newif\ifcomments\commentstrue %displays comments
%\newif\ifcomments\commentsfalse %so that comments do not display

\ifcomments
\newcommand{\authornote}[3]{\textcolor{#1}{[#3 ---#2]}}
\newcommand{\todo}[1]{\textcolor{red}{[TODO: #1]}}
\else
\newcommand{\authornote}[3]{}
\newcommand{\todo}[1]{}
\fi

\newcommand{\wss}[1]{\authornote{blue}{SS}{#1}} 
\newcommand{\plt}[1]{\authornote{magenta}{TPLT}{#1}} %For explanation of the template
\newcommand{\an}[1]{\authornote{cyan}{Author}{#1}}

%% Common Parts

\newcommand{\progname}{ProgName} % PUT YOUR PROGRAM NAME HERE
\newcommand{\authname}{Team \#, Team Name
\\ Student 1 name
\\ Student 2 name
\\ Student 3 name
\\ Student 4 name} % AUTHOR NAMES                  

\usepackage{hyperref}
    \hypersetup{colorlinks=true, linkcolor=blue, citecolor=blue, filecolor=blue,
                urlcolor=blue, unicode=false}
    \urlstyle{same}
                                


\begin{document}

\title{Software Requirements Specification for \progname: subtitle describing software} 
\author{\authname}
\date{\today}
	
\maketitle

~\newpage

\pagenumbering{roman}

\tableofcontents

~\newpage

\section*{Revision History}

\begin{tabularx}{\textwidth}{p{3cm}p{2cm}X}
\toprule {\textbf{Date}} & {\textbf{Version}} & {\textbf{Notes}}\\
\midrule
October 6, 2024 & 1.0 & Added Sections 6, 7, 8, 9, 10\\
October 6, 2024 & 1.0 & Added Sections 16, 17, 18, 19, 20\\
\bottomrule
\end{tabularx}

~\\

~\newpage
\section{Purpose of the Project}
\subsection{User Business}
\lips
\subsection{Goals of the Project}
\lips
\section{Stakeholders}
\subsection{Client}
\lips
\subsection{Customer}
\lips
\subsection{Other Stakeholders}
\lips
\subsection{Hands-On Users of the Project}
\lips
\subsection{Personas}
\lips
\subsection{Priorities Assigned to Users}
\lips
\subsection{User Participation}
\lips
\subsection{Maintenance Users and Service Technicians}
\lips

\section{Mandated Constraints}
\subsection{Solution Constraints}
\lips
\subsection{Implementation Environment of the Current System}
\lips
\subsection{Partner or Collaborative Applications}
\lips
\subsection{Off-the-Shelf Software}
\lips
\subsection{Anticipated Workplace Environment}
\lips
\subsection{Schedule Constraints}
\lips
\subsection{Budget Constraints}
\lips
\subsection{Enterprise Constraints}
\lips

\section{Naming Conventions and Terminology}
\subsection{Glossary of All Terms, Including Acronyms, Used by Stakeholders
involved in the Project}
\lips

\section{Relevant Facts And Assumptions}
\subsection{Relevant Facts}
\lips
\subsection{Business Rules}
\lips
\subsection{Assumptions}
\lips

\section{The Scope of the Work}
\subsection{The Current Situation}
The current state of dairy farming presents challenges in efficiently predicting
the health, productivity, and breeding outcomes of cattle. Farmers typically rely 
on historical records, but the analysis is done manually, often leading to 
reactive management. The existing systems do not utilize advanced technologies 
such as machine learning for predictive analytics. As a result, there is limited 
proactive management regarding milk production, breeding success, and herd longevity, 
which directly impacts farm profitability and sustainability.

The current solution environment lacks integration of large datasets from multiple 
sources, such as individual cow health records, breeding history, and environmental 
conditions, into a single system that can offer actionable predictions.
The current state of dairy farming presents challenges in efficiently predicting
the health, productivity, and breeding outcomes of cattle. Farmers typically rely 
on historical records, but the analysis is done manually, often leading to 
reactive management. The existing systems do not utilize advanced technologies 
such as machine learning for predictive analytics. As a result, there is limited 
proactive management regarding milk production, breeding success, and herd longevity, 
which directly impacts farm profitability and sustainability.

The current solution environment lacks integration of large datasets from multiple 
sources, such as individual cow health records, breeding history, and environmental 
conditions, into a single system that can offer actionable predictions.
\subsection{The Context of the Work}
This project aims to develop a machine learning model that will leverage 
historical herd data to predict important traits such as milk yield, breeding 
success rates, and the likelihood of a cow leaving the herd. This model will be 
integrated into a farm management system, providing farmers with actionable 
insights. The goal is to move from reactive to proactive herd management.

The model will use data such as the health, breed, and genetic history of both 
the mother and father to predict traits in calves. The software will be 
developed as part of a partnership with CATTLEytics Inc., ensuring seamless 
integration into their existing platform used by dairy farmers.
This project aims to develop a machine learning model that will leverage 
historical herd data to predict important traits such as milk yield, breeding 
success rates, and the likelihood of a cow leaving the herd. This model will be 
integrated into a farm management system, providing farmers with actionable 
insights. The goal is to move from reactive to proactive herd management.

The model will use data such as the health, breed, and genetic history of both 
the mother and father to predict traits in calves. The software will be 
developed as part of a partnership with CATTLEytics Inc., ensuring seamless 
integration into their existing platform used by dairy farmers.
\subsection{Work Partitioning}
The project will be divided into several key components:

\begin{enumerate}
    \item \textbf{Data Collection and Preprocessing:}
    \begin{itemize}
        \item Collection of historical data from existing systems, including 
        cow health records, breeding data, and productivity metrics.
        \item Cleaning and standardizing the data for input into the machine 
        learning model.
    \end{itemize}

    \item \textbf{Model Development:}
    \begin{itemize}
        \item Designing and implementing the machine learning model for trait 
        prediction (e.g., milk production, herd retention).
        \item Training and validating the model on historical datasets.
        \item Iterative testing and refinement.
    \end{itemize}

    \item \textbf{Integration:}
    \begin{itemize}
        \item Integrating the prediction model into the CATTLEytics farm 
        management system.
        \item Ensuring that outputs are presented in a user-friendly format 
        for farmers to make decisions.
    \end{itemize}

    \item \textbf{Testing and Validation:}
    \begin{itemize}
        \item Testing the software in real-world farm environments to validate 
        predictions and refine the user interface.
    \end{itemize}

    \item \textbf{Documentation and Training:}
    \begin{itemize}
        \item Providing clear documentation for users and training for farmers 
        to effectively use the system.
    \end{itemize}
\end{enumerate}
The project will be divided into several key components:

\begin{enumerate}
    \item \textbf{Data Collection and Preprocessing:}
    \begin{itemize}
        \item Collection of historical data from existing systems, including 
        cow health records, breeding data, and productivity metrics.
        \item Cleaning and standardizing the data for input into the machine 
        learning model.
    \end{itemize}

    \item \textbf{Model Development:}
    \begin{itemize}
        \item Designing and implementing the machine learning model for trait 
        prediction (e.g., milk production, herd retention).
        \item Training and validating the model on historical datasets.
        \item Iterative testing and refinement.
    \end{itemize}

    \item \textbf{Integration:}
    \begin{itemize}
        \item Integrating the prediction model into the CATTLEytics farm 
        management system.
        \item Ensuring that outputs are presented in a user-friendly format 
        for farmers to make decisions.
    \end{itemize}

    \item \textbf{Testing and Validation:}
    \begin{itemize}
        \item Testing the software in real-world farm environments to validate 
        predictions and refine the user interface.
    \end{itemize}

    \item \textbf{Documentation and Training:}
    \begin{itemize}
        \item Providing clear documentation for users and training for farmers 
        to effectively use the system.
    \end{itemize}
\end{enumerate}
\subsection{Specifying a Business Use Case (BUC)}

\textbf{Title:} Predicting Cow Traits for Optimized Herd Management

\textbf{Primary Actor:} Dairy farmer using the CATTLEytics system.

\textbf{Precondition:} The farmer has access to a herd management system 
integrated with the prediction model. Historical data on breeding, milk 
production, and herd turnover are available.

\textbf{Trigger:} The farmer initiates the model to predict the outcomes of a 
planned breeding or evaluates the likelihood of an existing cow leaving the herd.

\textbf{Main Success Scenario:}
\begin{enumerate}
    \item The farmer selects cows for breeding and inputs the necessary data 
    (e.g., parent traits).
    \item The system processes the input and returns predictions for milk yield 
    and herd retention likelihood.
    \item Based on the model's predictions, the farmer makes informed decisions 
    on breeding strategies or management actions to prevent herd loss.
\end{enumerate}

\textbf{Postconditions:}
\begin{itemize}
    \item The farmer has actionable insights to improve herd productivity and 
    manage herd turnover proactively.
\end{itemize}


\section{Business Data Model and Data Dictionary}
\subsection{Business Data Model}
This section will be completed once the relevant data model details are available.
This section will be completed once the relevant data model details are available.
\subsection{Data Dictionary}
This section will be completed once the relevant data model details are available.

% \section{The Scope of the Product}
% \subsection{Product Boundary}
% \lips
% \subsection{Product Use Case Table}
% \lips
% \subsection{Individual Product Use Cases (PUC's)}
% \lips
This section will be completed once the relevant data model details are available.

% \section{The Scope of the Product}
% \subsection{Product Boundary}
% \lips
% \subsection{Product Use Case Table}
% \lips
% \subsection{Individual Product Use Cases (PUC's)}
% \lips

\section{The Scope of the Product}

\subsection{Product Boundary}
The product's primary function is to predict cow traits such as milk production, 
breeding success, and herd retention based on parental data. The machine learning 
model will integrate into a platform like CATTLEytics, and its 
boundaries will include:

\begin{itemize}
    \item \textbf{Included}: The product will use historical data to generate 
    predictive insights on milk production, breeding success rates, and herd 
    retention likelihood. Farmers will be able to input relevant data to 
    receive predictions.
    \item \textbf{Not Included}: Real-time data collection and analysis, 
    live health monitoring, or any complex integrations with external devices 
    (e.g., IoT sensors).
\end{itemize}

This product will focus solely on predictive analytics based on historical data 
and will not handle aspects like data collection or advanced herd management 
automation beyond providing insights.

The product's primary function is to predict cow traits such as milk production, 
breeding success, and herd retention based on parental data. The machine learning 
model will integrate into a platform like CATTLEytics, and its 
boundaries will include:

\begin{itemize}
    \item \textbf{Included}: The product will use historical data to generate 
    predictive insights on milk production, breeding success rates, and herd 
    retention likelihood. Farmers will be able to input relevant data to 
    receive predictions.
    \item \textbf{Not Included}: Real-time data collection and analysis, 
    live health monitoring, or any complex integrations with external devices 
    (e.g., IoT sensors).
\end{itemize}

This product will focus solely on predictive analytics based on historical data 
and will not handle aspects like data collection or advanced herd management 
automation beyond providing insights.

\subsection{Product Use Case Table}
This table outlines the core use cases currently identified for the product, 
based on available information.



\begin{table}[h!]
\centering
\begin{tabularx}{\textwidth}{|c|l|l|X|}
    \hline
    \textbf{ID} & \textbf{Title} & \textbf{Actor} & \textbf{Description} \\
    \hline
    PUC1 & Predict Breeding Success & Dairy Farmer & Farmers input breeding 
    data to get predictions on the likelihood of successful breeding. \\
    \hline
    PUC2 & Forecast Milk Production & Dairy Farmer & Farmers input cow data 
    to get a prediction of future milk production. \\
    \hline
    PUC3 & Predict Herd Retention Likelihood & Dairy Farmer & Farmers receive 
    predictions on whether cows are likely to stay in or leave the herd. \\
    \hline
\end{tabularx}
\end{table}


This table outlines the core use cases currently identified for the product, 
based on available information.



\begin{table}[h!]
\centering
\begin{tabularx}{\textwidth}{|c|l|l|X|}
    \hline
    \textbf{ID} & \textbf{Title} & \textbf{Actor} & \textbf{Description} \\
    \hline
    PUC1 & Predict Breeding Success & Dairy Farmer & Farmers input breeding 
    data to get predictions on the likelihood of successful breeding. \\
    \hline
    PUC2 & Forecast Milk Production & Dairy Farmer & Farmers input cow data 
    to get a prediction of future milk production. \\
    \hline
    PUC3 & Predict Herd Retention Likelihood & Dairy Farmer & Farmers receive 
    predictions on whether cows are likely to stay in or leave the herd. \\
    \hline
\end{tabularx}
\end{table}


\subsection{Individual Product Use Cases (PUC's)}

\subsubsection{PUC1: Predict Breeding Success}
\begin{itemize}
    \item \textbf{Primary Actor}: Dairy Farmer
    \item \textbf{Preconditions}: Farmer has historical data about cow and 
    parental traits available for input.
    \item \textbf{Trigger}: The farmer initiates a request to predict 
    breeding success.
    \item \textbf{Main Success Scenario}:
    \begin{enumerate}
        \item The farmer enters relevant breeding data.
        \item The system processes the input using historical records.
        \item A prediction is generated on the likelihood of breeding success.
    \end{enumerate}
    \item \textbf{Postcondition}: The farmer gets an actionable prediction 
    to decide whether to proceed with the breeding.
\end{itemize}

\subsubsection{PUC2: Forecast Milk Production}
\begin{itemize}
    \item \textbf{Primary Actor}: Dairy Farmer
    \item \textbf{Preconditions}: Historical data for milk production and 
    parental traits is available for input.
    \item \textbf{Trigger}: The farmer requests a prediction for future milk 
    production.
    \item \textbf{Main Success Scenario}:
    \begin{enumerate}
        \item The farmer inputs the cow's data.
        \item The system processes the input data.
        \item A prediction on future milk production is generated.
    \end{enumerate}
    \item \textbf{Postcondition}: The farmer receives a prediction that helps 
    in planning milk yield expectations.
\end{itemize}

\subsubsection{PUC3: Predict Herd Retention Likelihood}
\begin{itemize}
    \item \textbf{Primary Actor}: Dairy Farmer
    \item \textbf{Preconditions}: Health and productivity data is available 
    for the cows in question.
    \item \textbf{Trigger}: The farmer requests predictions on herd retention 
    likelihood.
    \item \textbf{Main Success Scenario}:
    \begin{enumerate}
        \item The farmer selects a cow or group of cows for analysis.
        \item The system processes the available data.
        \item A prediction is generated on whether the cows are likely to stay 
        in or leave the herd.
    \end{enumerate}
    \item \textbf{Postcondition}: The farmer receives predictions to assist in 
    managing herd turnover.
\end{itemize}


\section{Functional Requirements}


\subsection{Functional Requirements}

\textbf{FR1: Predict Breeding Success}
\begin{itemize}
    \item \textbf{Description}: The system shall predict the likelihood of a 
    successful breeding event between two cows based on input data regarding 
    parental traits and historical breeding records.
    \item \textbf{Rationale}: This feature will help farmers make more informed 
    breeding decisions, improving breeding efficiency and reducing failures.
    \item \textbf{Fit Criterion}: The system will output a probability of 
    breeding success based on parental data, and this probability must be 
    verified by comparing predicted outcomes with actual breeding success over time.
\end{itemize}

\textbf{FR2: Forecast Milk Production}
\begin{itemize}
    \item \textbf{Description}: The system shall forecast the milk production 
    of a cow based on historical milk yield and parental traits.
    \item \textbf{Rationale}: Accurate predictions of future milk yield will 
    enable farmers to better plan for production and make decisions on herd management.
    \item \textbf{Fit Criterion}: The forecasted milk production must be within 
    10\% (to be determined) accuracy when compared to actual milk yield over a specified period.
\end{itemize}

\textbf{FR3: Predict Herd Retention Likelihood}
\begin{itemize}
    \item \textbf{Description}: The system shall predict the likelihood of 
    cows leaving the herd based on health records, productivity, and other 
    historical data.
    \item \textbf{Rationale}: This feature will enable farmers to proactively 
    manage their herd, reducing unexpected departures and improving herd stability.
    \item \textbf{Fit Criterion}: The system will provide a prediction score 
    (e.g., high, medium, low) for herd retention, which can be evaluated by 
    tracking actual herd retention over a six-month period.
\end{itemize}

\textbf{FR4: Data Input for Predictions}
\begin{itemize}
    \item \textbf{Description}: The system shall allow the farmer to input 
    relevant data, such as breeding records, milk production history, and 
    health records, into the prediction model.
    \item \textbf{Rationale}: To generate accurate predictions, the system 
    requires access to a range of historical data that can be inputted by the user.
    \item \textbf{Fit Criterion}: The input form must successfully accept and 
    validate required fields for at least 95\% of user inputs, with clear error 
    handling for missing or incorrect data.
\end{itemize}

\textbf{FR5: Report Generation}
\begin{itemize}
    \item \textbf{Description}: The system shall generate a report summarizing 
    predictions for breeding success, milk production, and herd retention for selected cows.
    \item \textbf{Rationale}: Farmers need a consolidated report that provides 
    actionable insights based on the predictions generated by the system.
    \item \textbf{Fit Criterion}: The system will generate reports that can be 
    exported to a PDF format and include all requested predictions in a structured layout.
\end{itemize}

\textbf{FR6: User Access Control}
\begin{itemize}
    \item \textbf{Description}: The system shall provide secure login and 
    role-based access control, ensuring that only authorized users can access 
    or modify the prediction data.
    \item \textbf{Rationale}: Farm management data is sensitive and should only 
    be accessible by authorized personnel.
    \item \textbf{Fit Criterion}: The system must enforce unique login credentials 
    for each user and restrict access based on roles (e.g., farmer, supervisor), 
    with at least 99\% reliability in access control enforcement.
\end{itemize}

\textbf{FR7: Integration with Farm Management System (if applicable)}
\begin{itemize}
    \item \textbf{Description}: The system shall be designed to integrate with 
    existing farm management platforms, such as CATTLEytics, allowing seamless 
    data exchange.
    \item \textbf{Rationale}: Integration with existing platforms will enable 
    the system to leverage historical data and provide predictions without 
    requiring manual data entry.
    \item \textbf{Fit Criterion}: The system should successfully exchange data 
    with the farm management platform 90\% of the time during testing, 
    without errors in data transmission.
\end{itemize}

\subsection{Formal Specification}

\textbf{Specification 1: Breeding Success Prediction}
\begin{itemize}
    \item \textbf{Description}: The system must be able to predict the likelihood 
    of breeding success between two cows based on historical data, such as 
    parental traits and previous breeding records.
    \item \textbf{Formal Specification}: \\
    Let \( X \) represent a breeding event. \\
    Let \( Y \) represent the set of all possible breeding events. \\
    Let \( P \) represent the predicted probability of success.
    
    \[
    \forall X \in Y : \text{Prediction}(X) \rightarrow P \in [0, 1]
    \]
    
    The system shall compute the probability \( P \) for each breeding event \( X \).
\end{itemize}

\textbf{Specification 2: Milk Production Forecast}
\begin{itemize}
    \item \textbf{Description}: The system shall forecast future milk production 
    for a given cow based on historical data of both the cow and its parents.
    \item \textbf{Formal Specification}: \\
    Let \( C \) represent a cow in the herd. \\
    Let \( Y \) represent historical milk production data. \\
    
    \[
    \forall C : \text{Forecast}(C, Y) \rightarrow \text{PredictedMilkProduction}(C)
    \]
    
    The system shall provide a forecast of future milk production for each 
    cow \( C \) based on input data \( Y \).
\end{itemize}

\textbf{Specification 3: Herd Retention Likelihood}
\begin{itemize}
    \item \textbf{Description}: The system must predict the likelihood of a cow 
    staying within or leaving the herd, based on its health, productivity, 
    and historical data.
    \item \textbf{Formal Specification}: \\
    Let \( H \) represent a cow's health record. \\
    Let \( P \) represent the predicted probability of retention.
    
    \[
    \forall H : \text{RetentionPrediction}(H) \rightarrow P \in [0, 1]
    \]
    
    The system shall compute the retention probability \( P \) for each cow 
    based on its health records and other historical data.
\end{itemize}

\textbf{Specification 4: Data Input Validation}
\begin{itemize}
    \item \textbf{Description}: The system must validate the input data for 
    cows and breeding events to ensure it is accurate and complete before 
    generating predictions.
    \item \textbf{Formal Specification}:
    Let $D$ represent the input data for a cow or breeding event.
    
    $\forall D : \text{InputValid}(D) = 
    \left\{
    \begin{array}{ll}
      \text{True} & \mname{if data passes validation checks} \\
      \text{False} & \mname{otherwise}
    \end{array}
    \right.$
    
    The system must ensure that all data $D$ is valid before processing 
    it for predictions.
\end{itemize}

\textbf{Specification 5: Report Generation (TBD)}
\begin{itemize}
    \item \textbf{Description}: The system must generate a report summarizing 
    predictions for breeding success, milk production, and herd retention likelihood.
    \item \textbf{Formal Specification}: \\
    Let \( R \) represent the report generated. \\
    \[
    \forall P, C : \text{GenerateReport}(P, C) \rightarrow R
    \]
    The system shall generate a report \( R \) based on the predictions \( P \) and input data \( C \).
\end{itemize}




\section{Look and Feel Requirements}
\subsection{Appearance Requirements}
\textbf{LFR1: Dashboard Display of Predictions}
\begin{itemize}
    \item \textbf{Description}: The system's dashboard shall present the 
    predicted cow traits (e.g., milk production, breeding success) in a 
    structured and organized manner, clearly showing individual predictions 
    for each cow.
    \item \textbf{Rationale}: Farmers need to quickly and easily interpret the 
    predictions without searching through large amounts of data. An organized 
    display ensures that all predictions can be understood at a glance.
    \item \textbf{Fit Criterion}: The system shall display predictions for 
    multiple cows in a table format, with clear labels for each trait, such 
    as milk production and herd retention likelihood.
\end{itemize}
\textbf{LFR2: Text Contrast for Readability}
\begin{itemize}
    \item \textbf{Description}: All text displayed on the system interface 
    shall use a high-contrast color scheme to ensure readability.
    \item \textbf{Rationale}: Farmers and users may access the system in 
    various lighting conditions. High contrast, such as black text on a 
    white background, will ensure clarity.
    \item \textbf{Fit Criterion}: The system shall use a high-contrast 
    color scheme for all text, ensuring that it meets standard readability 
    guidelines under different lighting conditions.
\end{itemize}

\subsection{Style Requirements}
\textbf{LFS1: Consistent Formatting for Input Fields}
\begin{itemize}
    \item \textbf{Description}: All data input fields, such as for entering 
    cow or parental traits, should follow a consistent format with clear labels 
    and input validation.
    \item \textbf{Rationale}: A consistent layout for input fields will 
    minimize errors and ensure ease of use when farmers input or update data.
    \item \textbf{Fit Criterion}: Input fields shall maintain a uniform 
    format, with clear labels and consistent spacing throughout the interface.
\end{itemize}
\textbf{LFS2: Minimalist Design for the Dashboard}
\begin{itemize}
    \item \textbf{Description}: The dashboard interface shall maintain a 
    clean and minimalist design, avoiding unnecessary clutter or decorative 
    elements.
    \item \textbf{Rationale}: A simplified interface will allow farmers to 
    focus on the essential data (predictions) without distractions, ensuring 
    ease of use.
    \item \textbf{Fit Criterion}: Over 80\% of users in a usability test 
    shall report that the dashboard is free from unnecessary elements and 
    easy to navigate.
\end{itemize}


\section{Usability and Humanity Requirements}
\subsection{Ease of Use Requirements}
\lips
\subsection{Personalization and Internationalization Requirements}
\lips
\subsection{Learning Requirements}
\lips
\subsection{Understandability and Politeness Requirements}
\lips
\subsection{Accessibility Requirements}
\lips

\section{Performance Requirements}
\subsection{Speed and Latency Requirements}
\lips
\subsection{Safety-Critical Requirements}
\lips
\subsection{Precision or Accuracy Requirements}
\lips
\subsection{Robustness or Fault-Tolerance Requirements}
\lips
\subsection{Capacity Requirements}
\lips
\subsection{Scalability or Extensibility Requirements}
\lips
\subsection{Longevity Requirements}
\lips

\section{Operational and Environmental Requirements}
\subsection{Expected Physical Environment}
\lips
\subsection{Wider Environment Requirements}
\lips
\subsection{Requirements for Interfacing with Adjacent Systems}
\lips
\subsection{Productization Requirements}
\lips
\subsection{Release Requirements}
\lips

\section{Maintainability and Support Requirements}
\subsection{Maintenance Requirements}
\lips
\subsection{Supportability Requirements}
\lips
\subsection{Adaptability Requirements}
\lips

\section{Security Requirements}
\subsection{Access Requirements}
\lips
\subsection{Integrity Requirements}
\lips
\subsection{Privacy Requirements}
\lips
\subsection{Audit Requirements}
\lips
\subsection{Immunity Requirements}
\lips

\section{Cultural Requirements}
\subsection{Cultural Requirements}
\begin{itemize}
	\item The primary language for the product will be English, tailored
	      specifically to Canadian dairy farmers. 
	\item All data and measurements will follow Canadian standards, including the
	      use of liters for milk production, kilograms for weight, and hectares
	      for land area Other relevant units such as Celsius for temperature and
	      metric tons for larger quantities may also be used
\end{itemize}

\section{Compliance Requirements}
\subsection{Legal Requirements}
\begin{itemize}
	\item The project must comply with the
	      \href{https://www.nfacc.ca/codes-of-practice/dairy-cattle}{Code of
	      Practice for the Care and Handling of Dairy Cattle}, which is a
	      government-regulated standard in Canada. This code outlines mandatory
	      guidelines for the ethical treatment, health, and welfare of dairy
	      cattle. Any management recommendations or actions suggested by the
	      machine learning model will align with these regulations to ensure
	      ethical practices in dairy farming.
	        
	\item The project must comply with
	      \href{https://laws-lois.justice.gc.ca/pdf/p-8.6.pdf}{PIPEDA} (Personal
	      Information Protection and Electronic Documents Act) for any personal
	      information related to dairy farmers or other individuals involved. This
	      includes the handling of contact details, financial information, and
	      other personally identifiable data.
\end{itemize}
\subsection{Standards Compliance Requirements}
\begin{itemize}
	\item There are no specific standards for collecting dairy farming data in
	      this project. All relevant aspects of data collection and handling are
	      already covered under Legal Requirements, specifically in compliance
	      with PIPEDA for managing sensitive information about dairy farmers, and
	      the Code of Practice for the Care and Handling of Dairy Cattle for
	      ensuring the welfare of the animals.
	\item For coding standards, the project will adhere to PEP8 to ensure
	      consistent and readable Python code. More information on PEP8 can be
	      found \href{https://peps.python.org/pep-0008/}{here}.
\end{itemize}

\section{Open Issues}
\begin{itemize}
	\item \textbf{Data Availability and Quality:} The accuracy of predictions will
	      heavily depend on the quality and completeness of the data obtained from
	      CATTLEytics and Lactanet. Inconsistent or missing data might affect the
	      performance of the model.
	      
	\item \textbf{Model Accuracy:} The machine learning model may need to be
	      fine-tuned multiple times to achieve high accuracy in predicting cow
	      traits. This requires testing with diverse datasets to ensure the model
	      generalizes well.
	      
	\item \textbf{User Interface Usability:} The graphical representation of the
	      family tree and predicted traits needs to be intuitive and user-friendly
	      for farmers with varying levels of technical skill. Determining the best
	      design and ensuring it meets users' needs could take time.
	      
	\item \textbf{Integration with CATTLEytics:} Seamlessly integrating the tool
	      into the existing CATTLEytics system without causing disruptions or
	      requiring major system changes could be technically challenging.
	      
	\item \textbf{Regulatory Compliance:} Ensuring that the predictions and
	      recommendations made by the model comply with Canadian regulations for
	      dairy farming (Code of Practice for the Care and Handling of Dairy
	      Cattle) will require thorough review and potential adjustments during
	      development.
	      
	\item \textbf{Model Interpretability:} Farmers may need clear explanations for
	      how predictions are made to trust and use the tool effectively. Ensuring
	      the model’s predictions are explainable is an open issue.
	      
	\item \textbf{Performance Considerations:} The tool needs to be efficient and
	      scalable, handling large amounts of data without significant lag or
	      performance issues, especially as it gets adopted by multiple farms.
\end{itemize}

\section{Off-the-Shelf Solutions}
\subsection{Ready-Made Products}
\begin{itemize}
	\item There are no fully ready-made products that address the predictive
	      capabilities being developed in this project. While tools like Lactanet
	      provide dairy farm data, they do not offer predictive models based on
	      genetic and health data. Lactanet data will be used primarily for
	      training the custom machine learning model.
\end{itemize}
\subsection{Reusable Components}
\begin{itemize}
	\item Machine learning libraries, such as PyTorch or TensorFlow, will be
	      utilized to develop the custom AI model for cow trait prediction.
	      Additionally, front-end libraries such as D3.js or React Tree
	      Visualization libraries could be considered for visualizing the
	      family-tree diagrams.
\end{itemize}
\subsection{Products That Can Be Copied}
\begin{itemize}
	\item There are no existing products to be copied for this project. However,
	      open-source family-tree visualization tools might serve as inspiration
	      for the graphical aspects of the project.
\end{itemize}

\section{New Problems}
\subsection{Effects on the Current Environment}
\begin{itemize}
	\item Introducing this system could change how farmers currently select sires
	      or evaluate herd performance. Some may resist adopting new technology
	      due to unfamiliarity.
\end{itemize}
\subsection{Effects on the Installed Systems}
\begin{itemize}
	\item The project will be integrated into the existing Cattleytics software,
	      which is already used to manage dairy farms. The machine learning tool
	      will act as an additional module within Cattleytics, allowing farmers to
	      visualize the family tree of cows and predict future traits based on
	      genetic data. Seamless integration with the current system will be
	      prioritized to ensure smooth adoption and ease of use.
\end{itemize}
\subsection{Potential User Problems}
\begin{itemize}
	\item Users may face difficulties interpreting complex AI model outputs, so
	      ensuring the tool’s recommendations are easy to understand is key.
\end{itemize}
\subsection{Limitations in the Anticipated Implementation Environment That May
Inhibit the New Product}
\begin{itemize}
	\item The tool will need to function effectively on standard farm computing
	      systems, which may have limited processing power or internet
	      connectivity.
\end{itemize}
\subsection{Follow-Up Problems}
\begin{itemize}
	\item Continuous updates may be needed to improve the model based on feedback
	      from farmers. Future updates may also need to address changes in farming
	      practices
\end{itemize}

\section{Tasks}
\subsection{Project Planning}
\lips
\subsection{Planning of the Development Phases}
\lips

\section{Migration to the New Product}
\subsection{Requirements for Migration to the New Product}
\lips
\subsection{Data That Has to be Modified or Translated for the New System}
\lips

\section{Costs}
\lips
\section{User Documentation and Training}
\subsection{User Documentation Requirements}
\lips
\subsection{Training Requirements}
\lips

\section{Waiting Room}
\lips

\section{Ideas for Solution}
\lips

\newpage{}
\section*{Appendix --- Reflection}

The information in this section will be used to evaluate the team members on the
graduate attribute of Lifelong Learning.  Please answer the following questions:

\begin{enumerate}
  \item What knowledge and skills will the team collectively need to acquire to
  successfully complete this capstone project?  Examples of possible knowledge
  to acquire include domain specific knowledge from the domain of your
  application, or software engineering knowledge, mechatronics knowledge or
  computer science knowledge.  Skills may be related to technology, or writing,
  or presentation, or team management, etc.  You should look to identify at
  least one item for each team member.
  \item For each of the knowledge areas and skills identified in the previous
  question, what are at least two approaches to acquiring the knowledge or
  mastering the skill?  Of the identified approaches, which will each team
  member pursue, and why did they make this choice?
\end{enumerate}

\end{document}